% Options for packages loaded elsewhere
\PassOptionsToPackage{unicode}{hyperref}
\PassOptionsToPackage{hyphens}{url}
\PassOptionsToPackage{dvipsnames,svgnames,x11names}{xcolor}
%
\documentclass[
  12pt,
  letterpaper,
  DIV=11,
  numbers=noendperiod]{scrartcl}

\usepackage{amsmath,amssymb}
\usepackage{lmodern}
\usepackage{iftex}
\ifPDFTeX
  \usepackage[T1]{fontenc}
  \usepackage[utf8]{inputenc}
  \usepackage{textcomp} % provide euro and other symbols
\else % if luatex or xetex
  \usepackage{unicode-math}
  \defaultfontfeatures{Scale=MatchLowercase}
  \defaultfontfeatures[\rmfamily]{Ligatures=TeX,Scale=1}
\fi
% Use upquote if available, for straight quotes in verbatim environments
\IfFileExists{upquote.sty}{\usepackage{upquote}}{}
\IfFileExists{microtype.sty}{% use microtype if available
  \usepackage[]{microtype}
  \UseMicrotypeSet[protrusion]{basicmath} % disable protrusion for tt fonts
}{}
\makeatletter
\@ifundefined{KOMAClassName}{% if non-KOMA class
  \IfFileExists{parskip.sty}{%
    \usepackage{parskip}
  }{% else
    \setlength{\parindent}{0pt}
    \setlength{\parskip}{6pt plus 2pt minus 1pt}}
}{% if KOMA class
  \KOMAoptions{parskip=half}}
\makeatother
\usepackage{xcolor}
\usepackage[margin=1in]{geometry}
\setlength{\emergencystretch}{3em} % prevent overfull lines
\setcounter{secnumdepth}{-\maxdimen} % remove section numbering
% Make \paragraph and \subparagraph free-standing
\ifx\paragraph\undefined\else
  \let\oldparagraph\paragraph
  \renewcommand{\paragraph}[1]{\oldparagraph{#1}\mbox{}}
\fi
\ifx\subparagraph\undefined\else
  \let\oldsubparagraph\subparagraph
  \renewcommand{\subparagraph}[1]{\oldsubparagraph{#1}\mbox{}}
\fi


\providecommand{\tightlist}{%
  \setlength{\itemsep}{0pt}\setlength{\parskip}{0pt}}\usepackage{longtable,booktabs,array}
\usepackage{calc} % for calculating minipage widths
% Correct order of tables after \paragraph or \subparagraph
\usepackage{etoolbox}
\makeatletter
\patchcmd\longtable{\par}{\if@noskipsec\mbox{}\fi\par}{}{}
\makeatother
% Allow footnotes in longtable head/foot
\IfFileExists{footnotehyper.sty}{\usepackage{footnotehyper}}{\usepackage{footnote}}
\makesavenoteenv{longtable}
\usepackage{graphicx}
\makeatletter
\def\maxwidth{\ifdim\Gin@nat@width>\linewidth\linewidth\else\Gin@nat@width\fi}
\def\maxheight{\ifdim\Gin@nat@height>\textheight\textheight\else\Gin@nat@height\fi}
\makeatother
% Scale images if necessary, so that they will not overflow the page
% margins by default, and it is still possible to overwrite the defaults
% using explicit options in \includegraphics[width, height, ...]{}
\setkeys{Gin}{width=\maxwidth,height=\maxheight,keepaspectratio}
% Set default figure placement to htbp
\makeatletter
\def\fps@figure{htbp}
\makeatother
\newlength{\cslhangindent}
\setlength{\cslhangindent}{1.5em}
\newlength{\csllabelwidth}
\setlength{\csllabelwidth}{3em}
\newlength{\cslentryspacingunit} % times entry-spacing
\setlength{\cslentryspacingunit}{\parskip}
\newenvironment{CSLReferences}[2] % #1 hanging-ident, #2 entry spacing
 {% don't indent paragraphs
  \setlength{\parindent}{0pt}
  % turn on hanging indent if param 1 is 1
  \ifodd #1
  \let\oldpar\par
  \def\par{\hangindent=\cslhangindent\oldpar}
  \fi
  % set entry spacing
  \setlength{\parskip}{#2\cslentryspacingunit}
 }%
 {}
\usepackage{calc}
\newcommand{\CSLBlock}[1]{#1\hfill\break}
\newcommand{\CSLLeftMargin}[1]{\parbox[t]{\csllabelwidth}{#1}}
\newcommand{\CSLRightInline}[1]{\parbox[t]{\linewidth - \csllabelwidth}{#1}\break}
\newcommand{\CSLIndent}[1]{\hspace{\cslhangindent}#1}

\usepackage{xr}
\usepackage[default]{sourcesanspro}
\usepackage{sourcecodepro}
\usepackage{lineno}
\linenumbers
\KOMAoption{captions}{tableheading}
\makeatletter
\makeatother
\makeatletter
\makeatother
\makeatletter
\@ifpackageloaded{caption}{}{\usepackage{caption}}
\AtBeginDocument{%
\ifdefined\contentsname
  \renewcommand*\contentsname{Table of contents}
\else
  \newcommand\contentsname{Table of contents}
\fi
\ifdefined\listfigurename
  \renewcommand*\listfigurename{List of Figures}
\else
  \newcommand\listfigurename{List of Figures}
\fi
\ifdefined\listtablename
  \renewcommand*\listtablename{List of Tables}
\else
  \newcommand\listtablename{List of Tables}
\fi
\ifdefined\figurename
  \renewcommand*\figurename{Fig.}
\else
  \newcommand\figurename{Fig.}
\fi
\ifdefined\tablename
  \renewcommand*\tablename{Table}
\else
  \newcommand\tablename{Table}
\fi
}
\@ifpackageloaded{float}{}{\usepackage{float}}
\floatstyle{ruled}
\@ifundefined{c@chapter}{\newfloat{codelisting}{h}{lop}}{\newfloat{codelisting}{h}{lop}[chapter]}
\floatname{codelisting}{Listing}
\newcommand*\listoflistings{\listof{codelisting}{List of Listings}}
\makeatother
\makeatletter
\@ifpackageloaded{caption}{}{\usepackage{caption}}
\@ifpackageloaded{subcaption}{}{\usepackage{subcaption}}
\makeatother
\makeatletter
\@ifpackageloaded{tcolorbox}{}{\usepackage[many]{tcolorbox}}
\makeatother
\makeatletter
\@ifundefined{shadecolor}{\definecolor{shadecolor}{rgb}{.97, .97, .97}}
\makeatother
\makeatletter
\makeatother
\ifLuaTeX
  \usepackage{selnolig}  % disable illegal ligatures
\fi
\IfFileExists{bookmark.sty}{\usepackage{bookmark}}{\usepackage{hyperref}}
\IfFileExists{xurl.sty}{\usepackage{xurl}}{} % add URL line breaks if available
\urlstyle{same} % disable monospaced font for URLs
\hypersetup{
  colorlinks=true,
  linkcolor={blue},
  filecolor={Maroon},
  citecolor={Blue},
  urlcolor={Blue},
  pdfcreator={LaTeX via pandoc}}

\author{}
\date{}

\begin{document}
\ifdefined\Shaded\renewenvironment{Shaded}{\begin{tcolorbox}[frame hidden, borderline west={3pt}{0pt}{shadecolor}, interior hidden, enhanced, breakable, sharp corners, boxrule=0pt]}{\end{tcolorbox}}\fi

\hypertarget{model}{%
\section{Model}\label{model}}

We modeled the seedling survival for the dry and rainy seasons
separately. Since the effect of tree neighbors on seedling survival is
nonlinear on a logistic scale (\protect\hyperlink{ref-Detto2019}{Detto
\emph{et al.} 2019}), we performed a grid-search for the scaling
parameter \emph{c} between 0 and 1 in 0.01 increments that maximized the
likelihood of the following survival model,

\[
\mathrm{logit}(p_i) = b_0 + b_1 Z_{1i}^c + b_2 Z_{2i}^c,
\]

where \(p_i\) is the individual survival probability in the \emph{i}th
census interval, and \(Z_1\) and \(Z_2\) are distance-weighted sums of
basal areas of conspecifics and heterospecifics respectively. We found
that \(c\) = 0.27 for the dry season and \(c\) = 0.24 for the rainy
season were the best estimates for our dataset.

We built Bayesian hierarchical models that include variation among
species in the effects of conspecific and heterospecific neighbours, and
rainfall on survival. Survival (\(s\)) of seedling record \emph{i} of
individual \emph{m} for species \emph{j} in census \emph{t} in plot
\emph{p} was modeled using the Bernoulli distribution (\(\mathcal{B}\)):

\[
s_{i,j,m,t,p} \sim \mathcal{B}(p_{i, j, m, t, p}),
\]

\[
\mathrm{logit}(p_{i,j,m,t,p}) = \boldsymbol{x_{i}} \cdot \boldsymbol{\beta_{j}} + \phi_p + \omega_t + \psi_m,
\]

where
\(\boldsymbol{\beta_{j}} = \left[\beta_{1,j}, \beta_{2,j}, \ldots, \beta_{K,j} \right]\)
is the coefficient row \emph{K}-vector for species \emph{j}, \emph{K} is
the number of predictors for an individual seedling,
\(\boldsymbol{x_i} = \left[x_{i,1},x _{i,2}, \ldots,x_{i,K} \right]\) is
the vector of predictors of size \emph{K} for an individual seedling,
\(\phi_p\) is the random effect for seedling plots, \(\omega_t\) is the
random effect for different census, and \(\psi_m\) is the random effect
for the repeated observations of the same individuals (note that
\(\cdot\) denotes dot product). The set of predictor variables
(\(\boldsymbol{x_i}\)) includes intercept, log of seedling heights,
rainfall, densities of conspecific (\emph{ConS}) and heterospecific
(\emph{HetS}) seedlings, densities of conspecific (\emph{ConA}) and
heterospecific (\emph{HetA}) trees that are scaled by 0.27 for the dry
season or 0.24 for the rainy season, and the interactions of rains with
\emph{ConS}, with \emph{HetS}, with \emph{ConA} and with \emph{HetA}.

In the species-level regression, the row vector of coefficients
(\(\boldsymbol{\beta_{1-K}}\)) of each species \emph{j} were assumed to
have a multivariate normal distribution through the Cholesky
factorization (\protect\hyperlink{ref-StanDevelopmentTeam2021}{Stan
Development Team 2021}),

\[
\boldsymbol{\beta_j} = \boldsymbol{\gamma_k} \cdot \boldsymbol{u_j} + \mathrm{diag}(\boldsymbol{\sigma})\cdot \boldsymbol{L} \cdot \boldsymbol{z},
\]

where
\(\boldsymbol{u_{j}} = \left[u_{1,j}, u_{2,j}, \ldots, u_{L,j} \right]\)
is the row vector of predictors of size \emph{L} for species \emph{j},
\emph{L} is the number of predictors for each species (i.e., the number
of traits including an intercept),
\(\boldsymbol{\gamma_k} = \left[\gamma_{k,1}, \gamma_{k,2}, \ldots, \gamma_{k,L} \right]\)
is the coefficient \emph{L}-vector for \emph{k}th predictor in the
individual-level regression, \(\mathrm{diag}(\boldsymbol{\sigma})\) is
the diagonal matrix with the diagonal vector of coefficient scales,
\(\boldsymbol{L}\) is the Cholesky factor of the original correlation
matrix which can be derived using a Cholesky decomposition for the
covariance matrix of the original multivariate normal distribution,
\(\boldsymbol{z}\) is a \emph{K} \(\times\) \emph{J} matrix of latent
Gaussian variables, and \emph{J} is the number of species. The set of
species-level predictor variables (\(\boldsymbol{u_j}\)) includes LDMC,
SDMC, LA, SLA, Chl, LT, \(\delta\)C\textsubscript{13}, C, N, and
\(\pi\)\textsubscript{tlp}. The row vector \(\gamma_{k,1}\) represents
average effects of the each individual-level predictor (e.g.,
\emph{ConS}) across species, whereas \(\gamma_{k, l}\) (\(l \ne 1\)),
represents the effects of the \emph{l}-th individual-level predictor
(e.g., SLA) on the variation in the strength of the each
individual-level predictor (e.g., variation in the strength of
\emph{ConS} among species). To allow comparisons among parameter
estimates, the individual-level predictors (\(\boldsymbol{x_i}\)) and
the species-level predictors (\(\boldsymbol{u_j}\)) were scaled to a
mean of 0 and standard deviation of 1 within each season and across
species, respectively.

Posterior distributions of all parameters were estimated using the
Hamiltonian Monte Carlo algorithm (HMC) implemented in Stan
(\protect\hyperlink{ref-Carpenter2017}{Carpenter \emph{et al.} 2017})
using the weakly-informative priors
(\protect\hyperlink{ref-Gelman2008}{Gelman \emph{et al.} 2008}).
Convergence of the posterior distribution was assessed with the
Gelman-Rubin statistic with a convergence threshold of 1.1 for all
parameters (\protect\hyperlink{ref-Gelman2013}{Gelman \emph{et al.}
2013}). All statistical analyses were conducted in R version 4.2.1
(\protect\hyperlink{ref-RCoreTeam2022}{R Core Team 2022}) using the R
package \emph{targets} version 0.12.1 for workflow management
(\protect\hyperlink{ref-Landau2021}{Landau 2021}).

\hypertarget{references}{%
\subsection*{References}\label{references}}
\addcontentsline{toc}{subsection}{References}

\hypertarget{refs}{}
\begin{CSLReferences}{1}{0}
\leavevmode\vadjust pre{\hypertarget{ref-Carpenter2017}{}}%
Carpenter, B., Gelman, A., Hoffman, M.D., Lee, D., Goodrich, B.,
Betancourt, M., \emph{et al.} (2017).
\href{https://doi.org/10.18637/jss.v076.i01}{Stan : {A Probabilistic
Programming Language}}. \emph{Journal of Statistical Software}, 76,
1--32.

\leavevmode\vadjust pre{\hypertarget{ref-Detto2019}{}}%
Detto, M., Visser, M.D., Wright, S.J. \& Pacala, S.W. (2019).
\href{https://doi.org/10.1111/ele.13372}{Bias in the detection of
negative density dependence in plant communities}. \emph{Ecology
Letters}, 22, 1923--1939.

\leavevmode\vadjust pre{\hypertarget{ref-Gelman2013}{}}%
Gelman, A., Carlin, J.B., Stern, H.S., Dunson, D.B., Vehtari, A. \&
Rubin, D.B. (2013). \emph{Bayesian {Data Analysis}, {Third Edition}}.
{Chapman \& Hall/CRC}, {Boca Raton, FL, USA.}

\leavevmode\vadjust pre{\hypertarget{ref-Gelman2008}{}}%
Gelman, A., Jakulin, A., Pittau, M.G. \& Su, Y.S. (2008).
\href{https://doi.org/10.1214/08-AOAS191}{A weakly informative default
prior distribution for logistic and other regression models}.
\emph{Annals of Applied Statistics}, 2, 1360--1383.

\leavevmode\vadjust pre{\hypertarget{ref-Landau2021}{}}%
Landau, W.M. (2021). \href{https://doi.org/10.21105/joss.02959}{The
targets {R} package: A dynamic {Make-like} function-oriented pipeline
toolkit for reproducibility and high-performance computing}.
\emph{Journal of Open Source Software}, 6, 2959.

\leavevmode\vadjust pre{\hypertarget{ref-RCoreTeam2022}{}}%
R Core Team. (2022). \emph{R: {A} language and environment for
statistical computing}. Manual. {R Foundation for Statistical
Computing}, {Vienna, Austria}.

\leavevmode\vadjust pre{\hypertarget{ref-StanDevelopmentTeam2021}{}}%
Stan Development Team. (2021). Stan user's guide v2.28 ({Accesed
December} 16 2021).

\end{CSLReferences}



\end{document}
