%%
% Copyright (c) 2017 - 2020, Pascal Wagler;
% Copyright (c) 2014 - 2020, John MacFarlane
%
% All rights reserved.
%
% Redistribution and use in source and binary forms, with or without
% modification, are permitted provided that the following conditions
% are met:
%
% - Redistributions of source code must retain the above copyright
% notice, this list of conditions and the following disclaimer.
%
% - Redistributions in binary form must reproduce the above copyright
% notice, this list of conditions and the following disclaimer in the
% documentation and/or other materials provided with the distribution.
%
% - Neither the name of John MacFarlane nor the names of other
% contributors may be used to endorse or promote products derived
% from this software without specific prior written permission.
%
% THIS SOFTWARE IS PROVIDED BY THE COPYRIGHT HOLDERS AND CONTRIBUTORS
% "AS IS" AND ANY EXPRESS OR IMPLIED WARRANTIES, INCLUDING, BUT NOT
% LIMITED TO, THE IMPLIED WARRANTIES OF MERCHANTABILITY AND FITNESS
% FOR A PARTICULAR PURPOSE ARE DISCLAIMED. IN NO EVENT SHALL THE
% COPYRIGHT OWNER OR CONTRIBUTORS BE LIABLE FOR ANY DIRECT, INDIRECT,
% INCIDENTAL, SPECIAL, EXEMPLARY, OR CONSEQUENTIAL DAMAGES (INCLUDING,
% BUT NOT LIMITED TO, PROCUREMENT OF SUBSTITUTE GOODS OR SERVICES;
% LOSS OF USE, DATA, OR PROFITS; OR BUSINESS INTERRUPTION) HOWEVER
% CAUSED AND ON ANY THEORY OF LIABILITY, WHETHER IN CONTRACT, STRICT
% LIABILITY, OR TORT (INCLUDING NEGLIGENCE OR OTHERWISE) ARISING IN
% ANY WAY OUT OF THE USE OF THIS SOFTWARE, EVEN IF ADVISED OF THE
% POSSIBILITY OF SUCH DAMAGE.
%%

%%
% This is the Eisvogel pandoc LaTeX template.
%
% For usage information and examples visit the official GitHub page:
% https://github.com/Wandmalfarbe/pandoc-latex-template
%%

\DeclareUnicodeCharacter{2212}{-}

% Options for packages loaded elsewhere
\PassOptionsToPackage{unicode}{hyperref}
\PassOptionsToPackage{hyphens}{url}
\PassOptionsToPackage{dvipsnames,svgnames*,x11names*,table}{xcolor}
%
\documentclass[
  12pt,
  a4paper,
,tablecaptionabove
]{scrartcl}
\usepackage{lmodern}
\usepackage{setspace}
\setstretch{1.2}
\usepackage{amssymb,amsmath}
\usepackage{ifxetex,ifluatex}
\ifnum 0\ifxetex 1\fi\ifluatex 1\fi=0 % if pdftex
  \usepackage[T1]{fontenc}
  \usepackage[utf8]{inputenc}
  \usepackage{textcomp} % provide euro and other symbols
\else % if luatex or xetex
  \usepackage{unicode-math}
  \defaultfontfeatures{Scale=MatchLowercase}
  \defaultfontfeatures[\rmfamily]{Ligatures=TeX,Scale=1}
\fi
% Use upquote if available, for straight quotes in verbatim environments
\IfFileExists{upquote.sty}{\usepackage{upquote}}{}
\IfFileExists{microtype.sty}{% use microtype if available
  \usepackage[]{microtype}
  \UseMicrotypeSet[protrusion]{basicmath} % disable protrusion for tt fonts
}{}
\makeatletter
\@ifundefined{KOMAClassName}{% if non-KOMA class
  \IfFileExists{parskip.sty}{%
    \usepackage{parskip}
  }{% else
    \setlength{\parindent}{0pt}
    \setlength{\parskip}{6pt plus 2pt minus 1pt}}
}{% if KOMA class
  \KOMAoptions{parskip=half}}
\makeatother
\usepackage{xcolor}
\definecolor{default-linkcolor}{HTML}{A50000}
\definecolor{default-filecolor}{HTML}{A50000}
\definecolor{default-citecolor}{HTML}{4077C0}
\definecolor{default-urlcolor}{HTML}{4077C0}
\IfFileExists{xurl.sty}{\usepackage{xurl}}{} % add URL line breaks if available
\IfFileExists{bookmark.sty}{\usepackage{bookmark}}{\usepackage{hyperref}}
\hypersetup{
  hidelinks,
  breaklinks=true,
  pdfcreator={LaTeX via pandoc with the Eisvogel template}}
\urlstyle{same} % disable monospaced font for URLs
\usepackage[margin=1in]{geometry}
% add backlinks to footnote references, cf. https://tex.stackexchange.com/questions/302266/make-footnote-clickable-both-ways
\usepackage{footnotebackref}
\setlength{\emergencystretch}{3em}  % prevent overfull lines
\providecommand{\tightlist}{%
  \setlength{\itemsep}{0pt}\setlength{\parskip}{0pt}}
\setcounter{secnumdepth}{-\maxdimen} % remove section numbering

% Make use of float-package and set default placement for figures to H.
% The option H means 'PUT IT HERE' (as  opposed to the standard h option which means 'You may put it here if you like').
\usepackage{float}
\floatplacement{figure}{H}

\usepackage{booktabs}
\usepackage{longtable}
\usepackage{array}
\usepackage{multirow}
\usepackage{wrapfig}
\usepackage{float}
\usepackage{colortbl}
\usepackage{pdflscape}
\usepackage{tabu}
\usepackage{threeparttable}
\usepackage{threeparttablex}
\usepackage[normalem]{ulem}
\usepackage{makecell}
\usepackage{xcolor}
\usepackage{lineno}
\linenumbers

\newlength{\cslhangindent}
\setlength{\cslhangindent}{1.5em}
\newlength{\csllabelwidth}
\setlength{\csllabelwidth}{3em}
\newenvironment{CSLReferences}[2] % #1 hanging-ident, #2 entry spacing
 {% don't indent paragraphs
  \setlength{\parindent}{0pt}
  % turn on hanging indent if param 1 is 1
  \ifodd #1 \everypar{\setlength{\hangindent}{\cslhangindent}}\ignorespaces\fi
  % set entry spacing
  \ifnum #2 > 0
  \setlength{\parskip}{#2\baselineskip}
  \fi
 }%
 {}
\usepackage{calc}
\newcommand{\CSLBlock}[1]{#1\hfill\break}
\newcommand{\CSLLeftMargin}[1]{\parbox[t]{\csllabelwidth}{#1}}
\newcommand{\CSLRightInline}[1]{\parbox[t]{\linewidth - \csllabelwidth}{#1}\break}
\newcommand{\CSLIndent}[1]{\hspace{\cslhangindent}#1}

\date{}


%%
%% added
%%

%
% language specification
%
% If no language is specified, use English as the default main document language.
%

\ifnum 0\ifxetex 1\fi\ifluatex 1\fi=0 % if pdftex
  \usepackage[shorthands=off,main=english]{babel}
\else
    % Workaround for bug in Polyglossia that breaks `\familydefault` when `\setmainlanguage` is used.
  % See https://github.com/Wandmalfarbe/pandoc-latex-template/issues/8
  % See https://github.com/reutenauer/polyglossia/issues/186
  % See https://github.com/reutenauer/polyglossia/issues/127
  \renewcommand*\familydefault{\sfdefault}
    % load polyglossia as late as possible as it *could* call bidi if RTL lang (e.g. Hebrew or Arabic)
  \usepackage{polyglossia}
  \setmainlanguage[]{english}
\fi



%
% for the background color of the title page
%

%
% break urls
%
\PassOptionsToPackage{hyphens}{url}

%
% When using babel or polyglossia with biblatex, loading csquotes is recommended
% to ensure that quoted texts are typeset according to the rules of your main language.
%
\usepackage{csquotes}

%
% captions
%
%\definecolor{caption-color}{HTML}{777777}
\definecolor{caption-color}{HTML}{37474F}
%\usepackage[font={stretch=1.2}, textfont={color=caption-color}, position=top, skip=4mm, labelfont=bf, singlelinecheck=false, justification=raggedright]{caption}
\usepackage[font={stretch=1}, textfont={color=caption-color}, position=top, skip=2mm, labelfont=bf, singlelinecheck=false, justification=raggedright]{caption}
\setcapindent{0em}

%
% blockquote
%
\definecolor{blockquote-border}{RGB}{221,221,221}
\definecolor{blockquote-text}{RGB}{119,119,119}
\usepackage{mdframed}
\newmdenv[rightline=false,bottomline=false,topline=false,linewidth=3pt,linecolor=blockquote-border,skipabove=\parskip]{customblockquote}
\renewenvironment{quote}{\begin{customblockquote}\list{}{\rightmargin=0em\leftmargin=0em}%
\item\relax\color{blockquote-text}\ignorespaces}{\unskip\unskip\endlist\end{customblockquote}}

%
% Source Sans Pro as the de­fault font fam­ily
% Source Code Pro for monospace text
%
% 'default' option sets the default
% font family to Source Sans Pro, not \sfdefault.
%
\ifnum 0\ifxetex 1\fi\ifluatex 1\fi=0 % if pdftex
    \usepackage[default]{sourcesanspro}
  \usepackage{sourcecodepro}
  %\usepackage{}
  \else % if not pdftex
    \usepackage[default]{sourcesanspro}
  \usepackage{sourcecodepro}
  %\usepackage{}

  % XeLaTeX specific adjustments for straight quotes: https://tex.stackexchange.com/a/354887
  % This issue is already fixed (see https://github.com/silkeh/latex-sourcecodepro/pull/5) but the
  % fix is still unreleased.
  % TODO: Remove this workaround when the new version of sourcecodepro is released on CTAN.
  \ifxetex
    \makeatletter
    \defaultfontfeatures[\ttfamily]
      { Numbers   = \sourcecodepro@figurestyle,
        Scale     = \SourceCodePro@scale,
        Extension = .otf }
    \setmonofont
      [ UprightFont    = *-\sourcecodepro@regstyle,
        ItalicFont     = *-\sourcecodepro@regstyle It,
        BoldFont       = *-\sourcecodepro@boldstyle,
        BoldItalicFont = *-\sourcecodepro@boldstyle It ]
      {SourceCodePro}
    \makeatother
  \fi
  \fi

%
% heading color
%
\definecolor{heading-color}{RGB}{40,40,40}
\addtokomafont{section}{\color{heading-color}}
% When using the classes report, scrreprt, book,
% scrbook or memoir, uncomment the following line.
%\addtokomafont{chapter}{\color{heading-color}}

%
% variables for title and author
%
\usepackage{titling}
\title{}
\author{}

%
% tables
%

%
% remove paragraph indention
%
\setlength{\parindent}{0pt}
\setlength{\parskip}{6pt plus 2pt minus 1pt}
\setlength{\emergencystretch}{3em}  % prevent overfull lines

%
%
% Listings
%
%


%
% header and footer
%
\usepackage{fancyhdr}

\fancypagestyle{eisvogel-header-footer}{
  \fancyhead{}
  \fancyfoot{}
  \lhead[]{}
  \chead[]{}
  \rhead[]{}
  %\lfoot[\thepage]{}
  \cfoot[]{}
  \cfoot[]{\thepage}
  \renewcommand{\headrulewidth}{0.0pt}
 % \renewcommand{\footrulewidth}{0.0pt}
 % \renewcommand{\headrulewidth}{0.4pt}
 % \renewcommand{\footrulewidth}{0.4pt}
}
\pagestyle{eisvogel-header-footer}

%%
%% end added
%%

\begin{document}

%%
%% begin titlepage
%%

%%
%% end titlepage
%%



\hypertarget{model}{%
\section{Model}\label{model}}

We modeled the seedling survival for the dry and rainy seasons
separately, in order to better understanding the seasonal dynamics of
seedling community.

\hypertarget{neighbor-densities}{%
\subsection{Neighbor densities}\label{neighbor-densities}}

Since the effect of adult neighbors on seedling survival is nonlinear in
the logistic scale (\protect\hyperlink{ref-Detto2019}{Detto et al.,
2019}), we performed a grid-search for the scaling parameter \(c\)
between 0 and 1 in 0.01 increments that maximized the likelihood of the
following survival model,

\[
\mathrm{logit}(p_i) = b_0 + b_1 Z_{1i}^c + b_2 Z_{2i}^c
\]

where \(p_i\) is the individual survival probability in the \emph{i}th
census interval, and \(Z_1\) and \(Z_2\) are distance-weighted sums of
basal areas of conspecifics and heterospecifics respectively. We found
that \(c\) = 0.24 for the dry season and \(c\) = 0.27 for the rainy
season were the best estimates for our dataset.

\hypertarget{survival-model}{%
\subsection{Survival model}\label{survival-model}}

We then build Bayesian hierarchical models that include variation among
species in the effects of conspecific and heterospecific neighbours, and
rainfall on survival. Survival (\(s\)) of seedling \emph{i} of
individual \emph{m} for species \emph{j} in census \emph{t} in plot
\emph{p} was modeled using the Bernoulli distribution (\(\mathcal{B}\))
:

\[
s_{i,j,m,t,p} \sim \mathcal{B}(p_{i, j, m, t, p})
\]

\[
\mathrm{logit}(p_{i,j,m,t,p}) = \boldsymbol{\beta_{j}} \cdot \boldsymbol{z_{i}} + \phi_p + \omega_t + \psi_m
\]

where
\(\boldsymbol{\beta_{j}} = \left[\beta_{j,1}, \beta_{j,2}, \ldots, \beta_{j,k} \right]\)
is the coefficient \emph{k}-vector for species \emph{j}, \emph{k} is the
number of predictors for an individual seedling,
\(\boldsymbol{z_i} = \left[z_{1,i}, z_{2,i}, \ldots, z_{k,i} \right]\)
is the row vector of predictors of size \emph{k} for an individual
seedling, \(\phi_p\) is the random effect for seedling plots,
\(\omega_t\) is the random effect for different census, and \(\psi_m\)
is the random effect for the repeated observations of the same
individuals (note that \(\cdot\) denotes dot product). The set of
predictor variables (\(z_{k,i}\)) includes intercept, log of seedling
heights, rainfall, densities of conspecific (CONS) and heterospecific
(HETS) seedlings, densities of conspecific (CONA) and heterospecific
(HETA) adult trees that are scaled by 0.24 for the dry season or 0.27
for the rainy season, and the interactions of rains with CONS, with
HETS, with CONA and with HETA.

In the species-level regression, the vector of coefficients
(\(\beta_{0-k}\)) of each species \emph{j} were assumed to have a
multivariate normal distribution (\(\mathcal{MVN}\));

\[
\boldsymbol{\beta_j} \sim  \mathcal{MVN}(\boldsymbol{x_j} \cdot \boldsymbol{\gamma_k}, {\boldsymbol \Sigma_{\beta}})
\]

where
\(\boldsymbol{x_{j}} = \left[x_{j,1}, x_{j,2}, \ldots, x_{j,l} \right]\)
is the vector of predictors of size \emph{l} for each species, \emph{l}
is the number of predictors for each species,
\(\boldsymbol{\gamma_i} = \left[\gamma_{1,k}, \gamma_{2,k}, \ldots, \gamma_{l,k} \right]\)
is the the coefficient \emph{l}-vector for species \emph{j} and
\(\boldsymbol{\Sigma_{\beta}}\) is the covariance matrix.

In the species-level regression, the vector of coefficients
(\(\beta_{0-k}\)) of each species \emph{j} were assumed to have a
multivariate normal distribution through the Cholesky factorization,

\[
\boldsymbol{\beta_j} = \boldsymbol{x_j} \cdot \boldsymbol{\gamma_k} + (\mathrm{diag}(\boldsymbol{\sigma})\cdot \boldsymbol{L} \cdot \boldsymbol{u})^\top
\]

where
\(\boldsymbol{x_{j}} = \left[x_{j,1}, x_{j,2}, \ldots, x_{j,l} \right]\)
is the vector of predictors of size \emph{l} for each species, \emph{l}
is the number of predictors for each species,
\(\boldsymbol{\gamma_i} = \left[\gamma_{1,k}, \gamma_{2,k}, \ldots, \gamma_{l,k} \right]\)
is the the coefficient \emph{l}-vector for species \emph{j} ,
\(\mathrm{diag}(\boldsymbol{\sigma})\) is the diagonal matrix with the
diagonal vector of coefficient scales, \(\boldsymbol{L}\) is the
Cholesky factor of the original correlation matrix which can be derived
using a Cholesky decomposition for the covariance matrix of the original
multivariate normal distribution, \(\boldsymbol{u}\) is a \(k \times j\)
matrix of latent Gaussian variable, and \(\top\) denotes the conjugate
transpose. The set of predictor variables (\(x_{j,l}\)) includes LDMC,
SDMC, LA, SLA, Chl, LT, C13, C, N, CN, and tlp (\textbf{need to edit
here according to a trait description section}).

Posterior distributions of all parameters were estimated using the
Hamiltonian Monte Carlo algorithm (HMC) implemented in Stan
(\protect\hyperlink{ref-Carpenter2017}{Carpenter et al., 2017}) using
the weakly-informative priors (\protect\hyperlink{ref-Gelman2008}{Gelman
et al., 2008}). See supplement X for more detail. The Stan code use to
fit models is available from Github at: XXXX. Convergence of the
posterior distribution was assessed with the Gelman-Rubin statistic with
a convergence threshold of 1.1 for all parameters
(\protect\hyperlink{ref-Gelman2013}{Gelman et al., 2013}).

\hypertarget{si-text}{%
\section{SI text}\label{si-text}}

\hypertarget{model-estimation}{%
\subsection{Model estimation}\label{model-estimation}}

We fitted the parameters using laten Gaussian variables for the
species-level coeffcient

\begin{align}
s_{i,j,m,t,p} &\sim \mathcal{B}(p_{i, j, m, t, p}) \\
\boldsymbol{\beta_j} &\sim \mathcal{MVN}(\boldsymbol{x_j} \cdot \boldsymbol{\gamma_k}, {\boldsymbol \Sigma_{\beta}})  \\
\gamma &\sim \mathcal{N}(0, 5) \\
L &\sim \mathrm{LkjCholesky}(2) \\
\tilde{\phi_p}, \tilde{\omega_t}, \tilde{\psi_m}, u &\sim \mathcal{N}(0, 1) \\
\tilde{\tau_{1-3}}, \tilde{\sigma_{1-3}} &\sim \mathcal{U}(0, \pi/2) \\
\beta &= x \cdot \gamma + (\mathrm{diag}(\sigma)\cdot L \cdot u)^\top \\
\tau_{1-3} &= 2.5 \times tan(\tilde{\tau_{1-3}}) \\
\sigma_{1-3} &= 2.5 \times tan(\tilde{\sigma_{1-3}}) \\
\phi_p &= \tau_1 \tilde{\phi_p} \\
\omega_t &= \tau_2 \tilde{\omega_t} \\
\psi_m &= \tau_3 \tilde{\psi_m} \\
\end{align}

parameter

\begin{align}
\mathrm{logit}(p_{i,j,m,t,p}) &= \boldsymbol{\beta_{j}} \cdot \boldsymbol{z_{i}} + \phi_p + \omega_t + \psi_m , \nonumber \\
\tau_{1-3} &= 2.5 \times tan(\tilde{\tau_{1-3}}), \nonumber \\
\sigma_{1-3} &= 2.5 \times tan(\tilde{\sigma_{1-3}}), \nonumber \\
\beta &= x \cdot \gamma + (\mathrm{diag}(\sigma)\cdot L \cdot u)^\top, \nonumber \\
\phi_p &= \tau_1 \tilde{\phi_p}, \nonumber \\
\omega_t &= \tau_2 \tilde{\omega_t}, \nonumber \\
\psi_m &= \tau_3 \tilde{\psi_m}
\end{align}

likelihood

\begin{align}
s_{i,j,m,t,p} &\sim \mathcal{B}(p_{i, j, m, t, p}) , \nonumber \\
\boldsymbol{\beta_j} &\sim \mathcal{MVN}(\boldsymbol{x_j} \cdot \boldsymbol{\gamma_k}, {\boldsymbol \Sigma_{\beta}}) , \nonumber \\
\tilde{\phi_p}, \tilde{\omega_t}, \tilde{\psi_m}, u &\sim \mathcal{N}(0, 1), \nonumber \\
\tilde{\tau_{1-3}}, \tilde{\sigma_{1-3}} &\sim \mathcal{U}(0, \pi/2), \nonumber \\
\gamma &\sim \mathcal{N}(0, 5), \nonumber \\
L &\sim \mathrm{LkjCholesky}(2)
\end{align}

The covariance matrix \(\boldsymbol \Sigma_{\beta}\) can be decomposed
as
\({\boldsymbol \Sigma} = \mathrm {diag}({\boldsymbol \sigma}){\boldsymbol \Omega}\mathrm {diag}({\boldsymbol \sigma}) = \mathrm {diag}({\boldsymbol \sigma})\boldsymbol{LL}^\top \mathrm {diag}({\boldsymbol \sigma})\)
using a Cholesky decomposition
(\protect\hyperlink{ref-Alvarez2014}{Alvarez et al., 2014}), where
\(\boldsymbol{L}\) is a Cholesky factor of a correlation matrix
(\(\boldsymbol \Omega\)), \(\boldsymbol{L}^\top\) is the conjugate
transpose of \(\boldsymbol{L}\), and \(\boldsymbol \sigma\) is the
vector of coefficient scales.

\hypertarget{references}{%
\subsection*{References}\label{references}}
\addcontentsline{toc}{subsection}{References}

\hypertarget{refs}{}
\begin{CSLReferences}{1}{0}
\leavevmode\vadjust pre{\hypertarget{ref-Alvarez2014}{}}%
Alvarez, I., J. Niemi, and M. Simpson. 2014.
\href{https://doi.org/10.1214/aos/1176348885}{Bayesian inference for a
covariance matrix}. \emph{arXiv preprint arXiv:1408.4050v2}: 1--12.

\leavevmode\vadjust pre{\hypertarget{ref-Carpenter2017}{}}%
Carpenter, B., A. Gelman, M. D. Hoffman, D. Lee, B. Goodrich, M.
Betancourt, M. Brubaker, et al. 2017.
\href{https://doi.org/10.18637/jss.v076.i01}{Stan : {A Probabilistic
Programming Language}}. \emph{Journal of Statistical Software} 76:
1--32.

\leavevmode\vadjust pre{\hypertarget{ref-Detto2019}{}}%
Detto, M., M. D. Visser, S. J. Wright, and S. W. Pacala. 2019.
\href{https://doi.org/10.1111/ele.13372}{Bias in the detection of
negative density dependence in plant communities}. \emph{Ecology
Letters} 22: 1923--1939.

\leavevmode\vadjust pre{\hypertarget{ref-Gelman2013}{}}%
Gelman, A., J. B. Carlin, H. S. Stern, D. B. Dunson, A. Vehtari, and D.
B. Rubin. 2013. Bayesian {Data Analysis}, {Third Edition}. {Taylor \&
Francis}.

\leavevmode\vadjust pre{\hypertarget{ref-Gelman2008}{}}%
Gelman, A., A. Jakulin, M. G. Pittau, and Y. S. Su. 2008.
\href{https://doi.org/10.1214/08-AOAS191}{A weakly informative default
prior distribution for logistic and other regression models}.
\emph{Annals of Applied Statistics} 2: 1360--1383.

\end{CSLReferences}

\end{document}
