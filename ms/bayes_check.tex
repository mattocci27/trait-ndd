% Options for packages loaded elsewhere
\PassOptionsToPackage{unicode}{hyperref}
\PassOptionsToPackage{hyphens}{url}
\PassOptionsToPackage{dvipsnames,svgnames,x11names}{xcolor}
%
\documentclass[
  11pt,
  letterpaper,
  DIV=11,
  numbers=noendperiod]{scrartcl}

\usepackage{amsmath,amssymb}
\usepackage{lmodern}
\usepackage{iftex}
\ifPDFTeX
  \usepackage[T1]{fontenc}
  \usepackage[utf8]{inputenc}
  \usepackage{textcomp} % provide euro and other symbols
\else % if luatex or xetex
  \usepackage{unicode-math}
  \defaultfontfeatures{Scale=MatchLowercase}
  \defaultfontfeatures[\rmfamily]{Ligatures=TeX,Scale=1}
\fi
% Use upquote if available, for straight quotes in verbatim environments
\IfFileExists{upquote.sty}{\usepackage{upquote}}{}
\IfFileExists{microtype.sty}{% use microtype if available
  \usepackage[]{microtype}
  \UseMicrotypeSet[protrusion]{basicmath} % disable protrusion for tt fonts
}{}
\makeatletter
\@ifundefined{KOMAClassName}{% if non-KOMA class
  \IfFileExists{parskip.sty}{%
    \usepackage{parskip}
  }{% else
    \setlength{\parindent}{0pt}
    \setlength{\parskip}{6pt plus 2pt minus 1pt}}
}{% if KOMA class
  \KOMAoptions{parskip=half}}
\makeatother
\usepackage{xcolor}
\setlength{\emergencystretch}{3em} % prevent overfull lines
\setcounter{secnumdepth}{-\maxdimen} % remove section numbering
% Make \paragraph and \subparagraph free-standing
\ifx\paragraph\undefined\else
  \let\oldparagraph\paragraph
  \renewcommand{\paragraph}[1]{\oldparagraph{#1}\mbox{}}
\fi
\ifx\subparagraph\undefined\else
  \let\oldsubparagraph\subparagraph
  \renewcommand{\subparagraph}[1]{\oldsubparagraph{#1}\mbox{}}
\fi

\usepackage{color}
\usepackage{fancyvrb}
\newcommand{\VerbBar}{|}
\newcommand{\VERB}{\Verb[commandchars=\\\{\}]}
\DefineVerbatimEnvironment{Highlighting}{Verbatim}{commandchars=\\\{\}}
% Add ',fontsize=\small' for more characters per line
\newenvironment{Shaded}{}{}
\newcommand{\AlertTok}[1]{\textcolor[rgb]{1.00,0.33,0.33}{\textbf{#1}}}
\newcommand{\AnnotationTok}[1]{\textcolor[rgb]{0.42,0.45,0.49}{#1}}
\newcommand{\AttributeTok}[1]{\textcolor[rgb]{0.84,0.23,0.29}{#1}}
\newcommand{\BaseNTok}[1]{\textcolor[rgb]{0.00,0.36,0.77}{#1}}
\newcommand{\BuiltInTok}[1]{\textcolor[rgb]{0.84,0.23,0.29}{#1}}
\newcommand{\CharTok}[1]{\textcolor[rgb]{0.01,0.18,0.38}{#1}}
\newcommand{\CommentTok}[1]{\textcolor[rgb]{0.42,0.45,0.49}{#1}}
\newcommand{\CommentVarTok}[1]{\textcolor[rgb]{0.42,0.45,0.49}{#1}}
\newcommand{\ConstantTok}[1]{\textcolor[rgb]{0.00,0.36,0.77}{#1}}
\newcommand{\ControlFlowTok}[1]{\textcolor[rgb]{0.84,0.23,0.29}{#1}}
\newcommand{\DataTypeTok}[1]{\textcolor[rgb]{0.84,0.23,0.29}{#1}}
\newcommand{\DecValTok}[1]{\textcolor[rgb]{0.00,0.36,0.77}{#1}}
\newcommand{\DocumentationTok}[1]{\textcolor[rgb]{0.42,0.45,0.49}{#1}}
\newcommand{\ErrorTok}[1]{\textcolor[rgb]{1.00,0.33,0.33}{\underline{#1}}}
\newcommand{\ExtensionTok}[1]{\textcolor[rgb]{0.84,0.23,0.29}{\textbf{#1}}}
\newcommand{\FloatTok}[1]{\textcolor[rgb]{0.00,0.36,0.77}{#1}}
\newcommand{\FunctionTok}[1]{\textcolor[rgb]{0.44,0.26,0.76}{#1}}
\newcommand{\ImportTok}[1]{\textcolor[rgb]{0.01,0.18,0.38}{#1}}
\newcommand{\InformationTok}[1]{\textcolor[rgb]{0.42,0.45,0.49}{#1}}
\newcommand{\KeywordTok}[1]{\textcolor[rgb]{0.84,0.23,0.29}{#1}}
\newcommand{\NormalTok}[1]{\textcolor[rgb]{0.14,0.16,0.18}{#1}}
\newcommand{\OperatorTok}[1]{\textcolor[rgb]{0.14,0.16,0.18}{#1}}
\newcommand{\OtherTok}[1]{\textcolor[rgb]{0.44,0.26,0.76}{#1}}
\newcommand{\PreprocessorTok}[1]{\textcolor[rgb]{0.84,0.23,0.29}{#1}}
\newcommand{\RegionMarkerTok}[1]{\textcolor[rgb]{0.42,0.45,0.49}{#1}}
\newcommand{\SpecialCharTok}[1]{\textcolor[rgb]{0.00,0.36,0.77}{#1}}
\newcommand{\SpecialStringTok}[1]{\textcolor[rgb]{0.01,0.18,0.38}{#1}}
\newcommand{\StringTok}[1]{\textcolor[rgb]{0.01,0.18,0.38}{#1}}
\newcommand{\VariableTok}[1]{\textcolor[rgb]{0.89,0.38,0.04}{#1}}
\newcommand{\VerbatimStringTok}[1]{\textcolor[rgb]{0.01,0.18,0.38}{#1}}
\newcommand{\WarningTok}[1]{\textcolor[rgb]{1.00,0.33,0.33}{#1}}

\providecommand{\tightlist}{%
  \setlength{\itemsep}{0pt}\setlength{\parskip}{0pt}}\usepackage{longtable,booktabs,array}
\usepackage{calc} % for calculating minipage widths
% Correct order of tables after \paragraph or \subparagraph
\usepackage{etoolbox}
\makeatletter
\patchcmd\longtable{\par}{\if@noskipsec\mbox{}\fi\par}{}{}
\makeatother
% Allow footnotes in longtable head/foot
\IfFileExists{footnotehyper.sty}{\usepackage{footnotehyper}}{\usepackage{footnote}}
\makesavenoteenv{longtable}
\usepackage{graphicx}
\makeatletter
\def\maxwidth{\ifdim\Gin@nat@width>\linewidth\linewidth\else\Gin@nat@width\fi}
\def\maxheight{\ifdim\Gin@nat@height>\textheight\textheight\else\Gin@nat@height\fi}
\makeatother
% Scale images if necessary, so that they will not overflow the page
% margins by default, and it is still possible to overwrite the defaults
% using explicit options in \includegraphics[width, height, ...]{}
\setkeys{Gin}{width=\maxwidth,height=\maxheight,keepaspectratio}
% Set default figure placement to htbp
\makeatletter
\def\fps@figure{htbp}
\makeatother
\newlength{\cslhangindent}
\setlength{\cslhangindent}{1.5em}
\newlength{\csllabelwidth}
\setlength{\csllabelwidth}{3em}
\newlength{\cslentryspacingunit} % times entry-spacing
\setlength{\cslentryspacingunit}{\parskip}
\newenvironment{CSLReferences}[2] % #1 hanging-ident, #2 entry spacing
 {% don't indent paragraphs
  \setlength{\parindent}{0pt}
  % turn on hanging indent if param 1 is 1
  \ifodd #1
  \let\oldpar\par
  \def\par{\hangindent=\cslhangindent\oldpar}
  \fi
  % set entry spacing
  \setlength{\parskip}{#2\cslentryspacingunit}
 }%
 {}
\usepackage{calc}
\newcommand{\CSLBlock}[1]{#1\hfill\break}
\newcommand{\CSLLeftMargin}[1]{\parbox[t]{\csllabelwidth}{#1}}
\newcommand{\CSLRightInline}[1]{\parbox[t]{\linewidth - \csllabelwidth}{#1}\break}
\newcommand{\CSLIndent}[1]{\hspace{\cslhangindent}#1}

\usepackage{booktabs}
\usepackage{longtable}
\usepackage{array}
\usepackage{multirow}
\usepackage{wrapfig}
\usepackage{float}
\usepackage{colortbl}
\usepackage{pdflscape}
\usepackage{tabu}
\usepackage{threeparttable}
\usepackage{threeparttablex}
\usepackage[normalem]{ulem}
\usepackage{makecell}
\usepackage{xcolor}
\usepackage{xr}
\usepackage[default]{sourcesanspro}
\usepackage{sourcecodepro}
\usepackage{fancyhdr}
\usepackage{fvextra}
\pagestyle{fancy}
\fancypagestyle{plain}{\pagestyle{fancy}}
\renewcommand{\headrulewidth}{0pt}
\fancyhead[RE,RO]{Song \textit{et al}. --Ecology Letters-- Appendix SX}
\DefineVerbatimEnvironment{Highlighting}{Verbatim}{breaklines,commandchars=\\\{\}}
\KOMAoption{captions}{tableheading}
\makeatletter
\makeatother
\makeatletter
\makeatother
\makeatletter
\@ifpackageloaded{caption}{}{\usepackage{caption}}
\AtBeginDocument{%
\ifdefined\contentsname
  \renewcommand*\contentsname{Table of contents}
\else
  \newcommand\contentsname{Table of contents}
\fi
\ifdefined\listfigurename
  \renewcommand*\listfigurename{List of Figures}
\else
  \newcommand\listfigurename{List of Figures}
\fi
\ifdefined\listtablename
  \renewcommand*\listtablename{List of Tables}
\else
  \newcommand\listtablename{List of Tables}
\fi
\ifdefined\figurename
  \renewcommand*\figurename{Figure}
\else
  \newcommand\figurename{Figure}
\fi
\ifdefined\tablename
  \renewcommand*\tablename{Table}
\else
  \newcommand\tablename{Table}
\fi
}
\@ifpackageloaded{float}{}{\usepackage{float}}
\floatstyle{ruled}
\@ifundefined{c@chapter}{\newfloat{codelisting}{h}{lop}}{\newfloat{codelisting}{h}{lop}[chapter]}
\floatname{codelisting}{Listing}
\newcommand*\listoflistings{\listof{codelisting}{List of Listings}}
\makeatother
\makeatletter
\@ifpackageloaded{caption}{}{\usepackage{caption}}
\@ifpackageloaded{subcaption}{}{\usepackage{subcaption}}
\makeatother
\makeatletter
\@ifpackageloaded{tcolorbox}{}{\usepackage[many]{tcolorbox}}
\makeatother
\makeatletter
\@ifundefined{shadecolor}{\definecolor{shadecolor}{rgb}{.97, .97, .97}}
\makeatother
\makeatletter
\makeatother
\ifLuaTeX
  \usepackage{selnolig}  % disable illegal ligatures
\fi
\IfFileExists{bookmark.sty}{\usepackage{bookmark}}{\usepackage{hyperref}}
\IfFileExists{xurl.sty}{\usepackage{xurl}}{} % add URL line breaks if available
\urlstyle{same} % disable monospaced font for URLs
\hypersetup{
  pdftitle={Appendix SX: Diagnostics for posterior distributions},
  colorlinks=true,
  linkcolor={blue},
  filecolor={Maroon},
  citecolor={Blue},
  urlcolor={Blue},
  pdfcreator={LaTeX via pandoc}}

\title{Appendix SX: Diagnostics for posterior distributions}
\author{}
\date{}

\begin{document}
\maketitle
\ifdefined\Shaded\renewenvironment{Shaded}{\begin{tcolorbox}[breakable, boxrule=0pt, borderline west={3pt}{0pt}{shadecolor}, frame hidden, sharp corners, interior hidden, enhanced]}{\end{tcolorbox}}\fi

\renewcommand*\contentsname{Table of contents}
{
\hypersetup{linkcolor=}
\setcounter{tocdepth}{3}
\tableofcontents
}
\newpage

Here we diagnose model convergence and divergent transitions for our
Bayesian model in the main text, then we check parameters that are
significantly different from zero. All codes are in R and Stan, and
workflow is managed with the R \texttt{targets} package
\href{https://github.com/ropensci/targets}{(https://github.com/ropensci/targets)}.
Data, codes, and computing environments to reproduce this manuscript
will be archived on zendo at https://doi.org/xxx and also available on
Github at \url{https://github.com/mattocci27/seedling-stan}.

\hypertarget{setup}{%
\section{Setup}\label{setup}}

First, we load R packages.

\begin{Shaded}
\begin{Highlighting}[]
\FunctionTok{library}\NormalTok{(tidyverse)}
\FunctionTok{library}\NormalTok{(kableExtra)}
\FunctionTok{library}\NormalTok{(here)}
\end{Highlighting}
\end{Shaded}

\begin{Shaded}
\begin{Highlighting}[]
\FunctionTok{source}\NormalTok{(}\FunctionTok{here}\NormalTok{(}\StringTok{"R"}\NormalTok{, }\StringTok{"stan.R"}\NormalTok{))}
\end{Highlighting}
\end{Shaded}

\hypertarget{data}{%
\section{Data}\label{data}}

All the data and results are in the \texttt{targets} cache, which can be
load using \texttt{tar\_load}.

\begin{itemize}
\item
  \texttt{dry\_each\_int\_s}: list object that contains data for dry
  seasons in our model
\item
  \texttt{wet\_each\_int\_s}: list object that contains data for rainy
  seasons in our model
\end{itemize}

\begin{Shaded}
\begin{Highlighting}[]
\NormalTok{targets}\SpecialCharTok{::}\FunctionTok{tar\_load}\NormalTok{(dry\_each\_int\_s)}
\NormalTok{targets}\SpecialCharTok{::}\FunctionTok{tar\_load}\NormalTok{(wet\_each\_int\_s)}
\end{Highlighting}
\end{Shaded}

\hypertarget{dry-season}{%
\subsection{Dry season}\label{dry-season}}

\begin{Shaded}
\begin{Highlighting}[]
\NormalTok{dry\_each\_int\_s }\SpecialCharTok{|\textgreater{}} \FunctionTok{str}\NormalTok{()}
\CommentTok{\#\textgreater{} List of 15}
\CommentTok{\#\textgreater{}  $ N     : int 29085}
\CommentTok{\#\textgreater{}  $ J     : int 76}
\CommentTok{\#\textgreater{}  $ K     : int 11}
\CommentTok{\#\textgreater{}  $ S     : int 384}
\CommentTok{\#\textgreater{}  $ T     : int 10}
\CommentTok{\#\textgreater{}  $ M     : int 9153}
\CommentTok{\#\textgreater{}  $ L     : int 11}
\CommentTok{\#\textgreater{}  $ cc    : num 0.27}
\CommentTok{\#\textgreater{}  $ suv   : num [1:29085] 0 1 0 1 1 0 0 0 1 0 ...}
\CommentTok{\#\textgreater{}  $ plot  : int [1:29085] 267 267 267 267 267 267 267 265 265 267 ...}
\CommentTok{\#\textgreater{}  $ census: int [1:29085] 1 1 1 7 7 1 1 8 6 1 ...}
\CommentTok{\#\textgreater{}  $ sp    : int [1:29085] 53 53 53 12 20 53 53 53 62 53 ...}
\CommentTok{\#\textgreater{}  $ tag   : int [1:29085] 6121 6185 6162 6138 6159 6122 6184 6070 6067 6165 ...}
\CommentTok{\#\textgreater{}  $ x     : num [1:29085, 1:11] 1 1 1 1 1 1 1 1 1 1 ...}
\CommentTok{\#\textgreater{}   ..{-} attr(*, "dimnames")=List of 2}
\CommentTok{\#\textgreater{}   .. ..$ : NULL}
\CommentTok{\#\textgreater{}   .. ..$ : chr [1:11] "int" "logh\_scaled" "cons\_scaled" "cona\_scaled\_c" ...}
\CommentTok{\#\textgreater{}  $ u     : num [1:11, 1:76] 1 1.337 1.175 0.87 {-}0.414 ...}
\CommentTok{\#\textgreater{}   ..{-} attr(*, "dimnames")=List of 2}
\CommentTok{\#\textgreater{}   .. ..$ : chr [1:11] "intercept" "ldmc" "sdmc" "chl" ...}
\CommentTok{\#\textgreater{}   .. ..$ : NULL}
\end{Highlighting}
\end{Shaded}

The elements in the object are as follows:

\begin{itemize}
\item
  \texttt{N}: number of individuals
\item
  \texttt{J}: number of species
\item
  \texttt{K}: number of individual-level predictors (e.g., ConS)
\item
  \texttt{S}: number of plots
\item
  \texttt{T}: number of censuses
\item
  \texttt{M}: number of the unique seedling individuals
\item
  \texttt{L}: number of species-level predictors (e.g., SLA)
\item
  \texttt{cc}: scaling parameter for conspecific and heterospecific tree
  densities
\item
  \texttt{suv}: survival data (0: dead, 1: alive)
\item
  \texttt{plot}: plot ID
\item
  \texttt{census}: census ID
\item
  \texttt{sp}: species ID
\item
  \texttt{tag}: seedling ID
\item
  \texttt{x}: N \(\times\) L matrix for individual-level predictors
\item
  \texttt{u}: N \(\times\) J matrix for species-level predictors
\end{itemize}

\hypertarget{rainy-season}{%
\subsection{Rainy season}\label{rainy-season}}

\begin{Shaded}
\begin{Highlighting}[]
\NormalTok{wet\_each\_int\_s }\SpecialCharTok{|\textgreater{}} \FunctionTok{str}\NormalTok{()}
\CommentTok{\#\textgreater{} List of 15}
\CommentTok{\#\textgreater{}  $ N     : int 26712}
\CommentTok{\#\textgreater{}  $ J     : int 76}
\CommentTok{\#\textgreater{}  $ K     : int 11}
\CommentTok{\#\textgreater{}  $ S     : int 384}
\CommentTok{\#\textgreater{}  $ T     : int 10}
\CommentTok{\#\textgreater{}  $ M     : int 7859}
\CommentTok{\#\textgreater{}  $ L     : int 11}
\CommentTok{\#\textgreater{}  $ cc    : num 0.24}
\CommentTok{\#\textgreater{}  $ suv   : num [1:26712] 1 0 1 0 1 1 0 0 0 0 ...}
\CommentTok{\#\textgreater{}  $ plot  : int [1:26712] 265 265 265 267 267 265 266 265 265 267 ...}
\CommentTok{\#\textgreater{}  $ census: int [1:26712] 7 8 7 8 8 7 8 8 8 8 ...}
\CommentTok{\#\textgreater{}  $ sp    : int [1:26712] 53 53 28 53 20 62 53 53 53 53 ...}
\CommentTok{\#\textgreater{}  $ tag   : int [1:26712] 4900 4896 4888 4946 4937 4887 4912 4894 4895 4945 ...}
\CommentTok{\#\textgreater{}  $ x     : num [1:26712, 1:11] 1 1 1 1 1 1 1 1 1 1 ...}
\CommentTok{\#\textgreater{}   ..{-} attr(*, "dimnames")=List of 2}
\CommentTok{\#\textgreater{}   .. ..$ : NULL}
\CommentTok{\#\textgreater{}   .. ..$ : chr [1:11] "int" "logh\_scaled" "cons\_scaled" "cona\_scaled\_c" ...}
\CommentTok{\#\textgreater{}  $ u     : num [1:11, 1:76] 1 1.337 1.175 0.87 {-}0.414 ...}
\CommentTok{\#\textgreater{}   ..{-} attr(*, "dimnames")=List of 2}
\CommentTok{\#\textgreater{}   .. ..$ : chr [1:11] "intercept" "ldmc" "sdmc" "chl" ...}
\CommentTok{\#\textgreater{}   .. ..$ : NULL}
\end{Highlighting}
\end{Shaded}

Details are same as in the dry seasons.

\hypertarget{model-stan-code}{%
\section{Model (Stan code)}\label{model-stan-code}}

Stan code for the multilevel logistic regression in the main text.

\begin{Shaded}
\begin{Highlighting}[]
\KeywordTok{data}\NormalTok{\{}
  \DataTypeTok{int}\NormalTok{\textless{}}\KeywordTok{lower}\NormalTok{=}\DecValTok{0}\NormalTok{\textgreater{} N; }\CommentTok{// number of sample}
  \DataTypeTok{int}\NormalTok{\textless{}}\KeywordTok{lower}\NormalTok{=}\DecValTok{1}\NormalTok{\textgreater{} J; }\CommentTok{// number of sp}
  \DataTypeTok{int}\NormalTok{\textless{}}\KeywordTok{lower}\NormalTok{=}\DecValTok{1}\NormalTok{\textgreater{} K; }\CommentTok{// number of tree{-}level preditor (i.e, CONS, HETS,...)}
  \DataTypeTok{int}\NormalTok{\textless{}}\KeywordTok{lower}\NormalTok{=}\DecValTok{1}\NormalTok{\textgreater{} L; }\CommentTok{// number of sp{-}level predictor (i.e., interecept and WP)}
  \DataTypeTok{int}\NormalTok{\textless{}}\KeywordTok{lower}\NormalTok{=}\DecValTok{1}\NormalTok{\textgreater{} M; }\CommentTok{// number of seedling individuals (tag)}
  \DataTypeTok{int}\NormalTok{\textless{}}\KeywordTok{lower}\NormalTok{=}\DecValTok{1}\NormalTok{\textgreater{} S; }\CommentTok{// number of site}
  \DataTypeTok{int}\NormalTok{\textless{}}\KeywordTok{lower}\NormalTok{=}\DecValTok{1}\NormalTok{\textgreater{} T; }\CommentTok{// number of census}
  \DataTypeTok{matrix}\NormalTok{[N, K] x; }\CommentTok{// tree{-}level predictor}
  \DataTypeTok{matrix}\NormalTok{[L, J] u; }\CommentTok{// sp{-}level predictor}
  \DataTypeTok{array}\NormalTok{[N] }\DataTypeTok{int}\NormalTok{\textless{}}\KeywordTok{lower}\NormalTok{=}\DecValTok{0}\NormalTok{,}\KeywordTok{upper}\NormalTok{=}\DecValTok{1}\NormalTok{\textgreater{} suv; }\CommentTok{// 1 or 0}
  \DataTypeTok{array}\NormalTok{[N] }\DataTypeTok{int}\NormalTok{\textless{}}\KeywordTok{lower}\NormalTok{=}\DecValTok{1}\NormalTok{,}\KeywordTok{upper}\NormalTok{=J\textgreater{} sp; }\CommentTok{// integer}
  \DataTypeTok{array}\NormalTok{[N] }\DataTypeTok{int}\NormalTok{\textless{}}\KeywordTok{lower}\NormalTok{=}\DecValTok{1}\NormalTok{,}\KeywordTok{upper}\NormalTok{=S\textgreater{} plot; }\CommentTok{// integer}
  \DataTypeTok{array}\NormalTok{[N] }\DataTypeTok{int}\NormalTok{\textless{}}\KeywordTok{lower}\NormalTok{=}\DecValTok{1}\NormalTok{,}\KeywordTok{upper}\NormalTok{=T\textgreater{} census; }\CommentTok{// integer}
  \DataTypeTok{array}\NormalTok{[N] }\DataTypeTok{int}\NormalTok{\textless{}}\KeywordTok{lower}\NormalTok{=}\DecValTok{1}\NormalTok{\textgreater{} tag; }\CommentTok{// integer}
\NormalTok{\}}

\KeywordTok{parameters}\NormalTok{\{}
  \DataTypeTok{matrix}\NormalTok{[K, L] gamma;}
  \DataTypeTok{matrix}\NormalTok{[K, J] z;}
  \DataTypeTok{vector}\NormalTok{[S] phi\_raw;}
  \DataTypeTok{vector}\NormalTok{[T] xi\_raw;}
  \DataTypeTok{vector}\NormalTok{[M] psi\_raw;}
  \DataTypeTok{cholesky\_factor\_corr}\NormalTok{[K] L\_Omega;}
  \DataTypeTok{vector}\NormalTok{\textless{}}\KeywordTok{lower}\NormalTok{=}\DecValTok{0}\NormalTok{,}\KeywordTok{upper}\NormalTok{=pi()/}\DecValTok{2}\NormalTok{\textgreater{}[K] tau\_unif;}
  \DataTypeTok{vector}\NormalTok{\textless{}}\KeywordTok{lower}\NormalTok{=}\DecValTok{0}\NormalTok{,}\KeywordTok{upper}\NormalTok{=pi()/}\DecValTok{2}\NormalTok{\textgreater{}[}\DecValTok{3}\NormalTok{] sig\_unif;}
\NormalTok{\}}

\KeywordTok{transformed parameters}\NormalTok{\{}
  \DataTypeTok{matrix}\NormalTok{[K, J] beta;}
  \DataTypeTok{vector}\NormalTok{\textless{}}\KeywordTok{lower}\NormalTok{=}\DecValTok{0}\NormalTok{\textgreater{}[K] tau;}
  \DataTypeTok{vector}\NormalTok{\textless{}}\KeywordTok{lower}\NormalTok{=}\DecValTok{0}\NormalTok{\textgreater{}[}\DecValTok{3}\NormalTok{] sig;}
  \DataTypeTok{vector}\NormalTok{[S] phi;}
  \DataTypeTok{vector}\NormalTok{[T] xi;}
  \DataTypeTok{vector}\NormalTok{[M] psi;}
  \ControlFlowTok{for}\NormalTok{ (k }\ControlFlowTok{in} \DecValTok{1}\NormalTok{:K) tau[k] = }\FloatTok{2.5}\NormalTok{ * tan(tau\_unif[k]);}
  \ControlFlowTok{for}\NormalTok{ (i }\ControlFlowTok{in} \DecValTok{1}\NormalTok{:}\DecValTok{3}\NormalTok{) sig[i] = }\FloatTok{2.5}\NormalTok{ * tan(sig\_unif[i]);}
\NormalTok{  beta = gamma * u + diag\_pre\_multiply(tau, L\_Omega) * z;}
\NormalTok{  phi = phi\_raw * sig[}\DecValTok{1}\NormalTok{];}
\NormalTok{  xi = xi\_raw * sig[}\DecValTok{2}\NormalTok{];}
\NormalTok{  psi = psi\_raw * sig[}\DecValTok{3}\NormalTok{];}
\NormalTok{\}}

\KeywordTok{model}\NormalTok{ \{}
  \DataTypeTok{vector}\NormalTok{[N] mu;}
\NormalTok{  to\_vector(z) \textasciitilde{} std\_normal();}
\NormalTok{  to\_vector(phi\_raw) \textasciitilde{} std\_normal();}
\NormalTok{  to\_vector(xi\_raw) \textasciitilde{} std\_normal();}
\NormalTok{  to\_vector(psi\_raw) \textasciitilde{} std\_normal();}
\NormalTok{  L\_Omega \textasciitilde{} lkj\_corr\_cholesky(}\DecValTok{2}\NormalTok{);}
\NormalTok{  to\_vector(gamma) \textasciitilde{} normal(}\DecValTok{0}\NormalTok{, }\DecValTok{5}\NormalTok{);}
  \ControlFlowTok{for}\NormalTok{ (n }\ControlFlowTok{in} \DecValTok{1}\NormalTok{:N) \{}
\NormalTok{    mu[n] = x[n, ] * beta[, sp[n]];}
\NormalTok{  \}}
\NormalTok{  suv \textasciitilde{} bernoulli\_logit(mu + phi[plot] + xi[census] + psi[tag]);}
\NormalTok{\}}

\KeywordTok{generated quantities}\NormalTok{ \{}
  \DataTypeTok{vector}\NormalTok{[N] log\_lik;}
  \DataTypeTok{corr\_matrix}\NormalTok{[K] Omega;}
\NormalTok{  Omega = multiply\_lower\_tri\_self\_transpose(L\_Omega);}
  \ControlFlowTok{for}\NormalTok{ (n }\ControlFlowTok{in} \DecValTok{1}\NormalTok{:N) \{}
\NormalTok{    log\_lik[n] = bernoulli\_logit\_lpmf(suv[n] | x[n, ] * beta[, sp[n]] +}
\NormalTok{      phi[plot[n]] + xi[census[n]] + psi[tag[n]]);}
\NormalTok{  \}}
\NormalTok{\}}
\end{Highlighting}
\end{Shaded}

\hypertarget{model-validation}{%
\section{Model validation}\label{model-validation}}

Posterior distributions of all parameters were estimated using the
Hamiltonian Monte Carlo algorithm (HMC) implemented in Stan
(\protect\hyperlink{ref-Carpenter2017}{Carpenter \emph{et al.} 2017};
\protect\hyperlink{ref-StanDevelopmentTeam2022}{Stan Development Team
2022}). Posterior estimates were obtained from 4 independent chains of
1,000 iteraions after a warmup of 1,000 iterations. Convergence of the
posterior distribution was assessed with the Gelman-Rubin statistic
(\protect\hyperlink{ref-Gelman2013}{Gelman \emph{et al.} 2013}) with a
convergence threshold of 1.1 for all parameters as following.

\hypertarget{convergence}{%
\subsection{Convergence}\label{convergence}}

\begin{itemize}
\item
  \texttt{fit\_9\_dry\_each\_int\_s\_summary\_model\_ind}: summaries of
  posterior distributions of all the parameters for dry seasons
\item
  \texttt{fit\_10\_wet\_each\_int\_s\_summary\_model\_ind}: summaries of
  posterior distributions of all the parameters for rainy seasons
\end{itemize}

\begin{Shaded}
\begin{Highlighting}[]
\NormalTok{targets}\SpecialCharTok{::}\FunctionTok{tar\_load}\NormalTok{(fit\_9\_dry\_each\_int\_s\_summary\_model\_ind)}
\NormalTok{targets}\SpecialCharTok{::}\FunctionTok{tar\_load}\NormalTok{(fit\_10\_wet\_each\_int\_s\_summary\_model\_ind)}
\end{Highlighting}
\end{Shaded}

There were no parameters with Rhat \textgreater{} 1.1, which indicates
both models have converged.

\begin{Shaded}
\begin{Highlighting}[]
\NormalTok{fit\_9\_dry\_each\_int\_s\_summary\_model\_ind }\SpecialCharTok{|\textgreater{}} \FunctionTok{filter}\NormalTok{(rhat }\SpecialCharTok{\textgreater{}} \FloatTok{1.1}\NormalTok{)}
\CommentTok{\#\textgreater{} \# A tibble: 0 x 11}
\CommentTok{\#\textgreater{} \# ... with 11 variables: variable \textless{}chr\textgreater{}, mean \textless{}dbl\textgreater{}, median \textless{}dbl\textgreater{}, sd \textless{}dbl\textgreater{},}
\CommentTok{\#\textgreater{} \#   mad \textless{}dbl\textgreater{}, q5 \textless{}dbl\textgreater{}, q95 \textless{}dbl\textgreater{}, rhat \textless{}dbl\textgreater{}, ess\_bulk \textless{}dbl\textgreater{}, ess\_tail \textless{}dbl\textgreater{},}
\CommentTok{\#\textgreater{} \#   .join\_data \textless{}dbl\textgreater{}}
\NormalTok{fit\_10\_wet\_each\_int\_s\_summary\_model\_ind }\SpecialCharTok{|\textgreater{}} \FunctionTok{filter}\NormalTok{(rhat }\SpecialCharTok{\textgreater{}} \FloatTok{1.1}\NormalTok{)}
\CommentTok{\#\textgreater{} \# A tibble: 0 x 11}
\CommentTok{\#\textgreater{} \# ... with 11 variables: variable \textless{}chr\textgreater{}, mean \textless{}dbl\textgreater{}, median \textless{}dbl\textgreater{}, sd \textless{}dbl\textgreater{},}
\CommentTok{\#\textgreater{} \#   mad \textless{}dbl\textgreater{}, q5 \textless{}dbl\textgreater{}, q95 \textless{}dbl\textgreater{}, rhat \textless{}dbl\textgreater{}, ess\_bulk \textless{}dbl\textgreater{}, ess\_tail \textless{}dbl\textgreater{},}
\CommentTok{\#\textgreater{} \#   .join\_data \textless{}dbl\textgreater{}}
\end{Highlighting}
\end{Shaded}

\hypertarget{divergent-transitions}{%
\subsection{Divergent transitions}\label{divergent-transitions}}

\begin{Shaded}
\begin{Highlighting}[]
\NormalTok{targets}\SpecialCharTok{::}\FunctionTok{tar\_load}\NormalTok{(fit\_9\_dry\_each\_int\_s\_diagnostics\_model\_ind)}
\NormalTok{targets}\SpecialCharTok{::}\FunctionTok{tar\_load}\NormalTok{(fit\_10\_wet\_each\_int\_s\_diagnostics\_model\_ind)}
\end{Highlighting}
\end{Shaded}

\begin{itemize}
\item
  \texttt{fit\_9\_dry\_each\_int\_s\_diagnostics\_model\_ind}: the
  number of divergences during HMC for dry seasons
\item
  \texttt{fit\_10\_wet\_each\_int\_s\_diagnostics\_model\_ind}: the
  number of divergences during HMC for rainy seasons
\end{itemize}

When the number of divergences is too large, posterior distributions are
not reliable (\protect\hyperlink{ref-Betancourt2016}{Betancourt 2016}).

Here is the code to check the number of divergences:

\begin{Shaded}
\begin{Highlighting}[]
\NormalTok{div\_check}
\CommentTok{\#\textgreater{} function (diags) }
\CommentTok{\#\textgreater{} \{}
\CommentTok{\#\textgreater{}     n1 \textless{}{-} nrow(filter(diags, divergent\_\_ == 1))}
\CommentTok{\#\textgreater{}     n2 \textless{}{-} nrow(diags)}
\CommentTok{\#\textgreater{}     print(paste(n1, "of", n2, "iterations ended with a divergence", }
\CommentTok{\#\textgreater{}         n1/n2 * 100, "\%"))}
\CommentTok{\#\textgreater{} \}}
\end{Highlighting}
\end{Shaded}

There were no divergent transitions, suggesting that HMC trajectory
tracked the true trajectory
(\protect\hyperlink{ref-StanDevelopmentTeam2022}{Stan Development Team
2022}).

\begin{Shaded}
\begin{Highlighting}[]
\FunctionTok{div\_check}\NormalTok{(fit\_9\_dry\_each\_int\_s\_diagnostics\_model\_ind)}
\CommentTok{\#\textgreater{} [1] "0 of 4000 iterations ended with a divergence 0 \%"}
\FunctionTok{div\_check}\NormalTok{(fit\_10\_wet\_each\_int\_s\_diagnostics\_model\_ind)}
\CommentTok{\#\textgreater{} [1] "0 of 4000 iterations ended with a divergence 0 \%"}
\end{Highlighting}
\end{Shaded}

\hypertarget{tables}{%
\section{Tables}\label{tables}}

\hypertarget{gamma-coefficients-for-dry-seasons}{%
\subsection{\texorpdfstring{\(\gamma\) coefficients for dry
seasons}{\textbackslash gamma coefficients for dry seasons}}\label{gamma-coefficients-for-dry-seasons}}

\begin{itemize}
\tightlist
\item
  \texttt{fit9\_gamma}: \(\gamma\) values of all the parameters for dry
  seasons
\end{itemize}

\begin{Shaded}
\begin{Highlighting}[]
\NormalTok{targets}\SpecialCharTok{::}\FunctionTok{tar\_load}\NormalTok{(fit9\_gamma)}
\NormalTok{fit9\_gamma[}\DecValTok{2}\SpecialCharTok{:}\DecValTok{6}\NormalTok{] }\OtherTok{\textless{}{-}} \FunctionTok{round}\NormalTok{(fit9\_gamma[}\DecValTok{2}\SpecialCharTok{:}\DecValTok{6}\NormalTok{], }\DecValTok{3}\NormalTok{)}
\end{Highlighting}
\end{Shaded}

This is the list of \(\gamma\) parameters that were significantly
different from zero (i.e., the 95\% credible intervals did not include
zero). The full list is available as Table S1 in a separate file.

\begin{Shaded}
\begin{Highlighting}[]
\NormalTok{fit9\_gamma }\SpecialCharTok{|\textgreater{}}
  \FunctionTok{filter}\NormalTok{(q2\_5 }\SpecialCharTok{*}\NormalTok{ q97\_5 }\SpecialCharTok{\textgreater{}} \DecValTok{0}\NormalTok{) }\SpecialCharTok{|\textgreater{}}
  \FunctionTok{arrange}\NormalTok{(para) }\SpecialCharTok{|\textgreater{}}
  \FunctionTok{kbl}\NormalTok{(}\AttributeTok{booktabs =} \ConstantTok{TRUE}\NormalTok{, }\AttributeTok{longtable =} \ConstantTok{TRUE}\NormalTok{, }\AttributeTok{format =} \StringTok{"latex"}\NormalTok{) }\SpecialCharTok{|\textgreater{}}
  \FunctionTok{kable\_styling}\NormalTok{(}\AttributeTok{latex\_options =} \FunctionTok{c}\NormalTok{(}\StringTok{"striped"}\NormalTok{, }\StringTok{"repeat\_header"}\NormalTok{))}
\end{Highlighting}
\end{Shaded}

\begin{longtable}[t]{lrrrrrll}
\toprule
para & mean\_ & q2\_5 & q5 & q95 & q97\_5 & pred\_name & trait\_name\\
\midrule
\endfirsthead
\multicolumn{8}{@{}l}{\textit{(continued)}}\\
\toprule
para & mean\_ & q2\_5 & q5 & q95 & q97\_5 & pred\_name & trait\_name\\
\midrule
\endhead

\endfoot
\bottomrule
\endlastfoot
\cellcolor{gray!6}{gamma\_1\_1} & \cellcolor{gray!6}{2.693} & \cellcolor{gray!6}{1.818} & \cellcolor{gray!6}{1.957} & \cellcolor{gray!6}{3.302} & \cellcolor{gray!6}{3.606} & \cellcolor{gray!6}{int} & \cellcolor{gray!6}{intercept}\\
gamma\_1\_6 & -1.148 & -2.135 & -1.977 & -0.506 & -0.143 & int & c\_mass\\
\cellcolor{gray!6}{gamma\_2\_1} & \cellcolor{gray!6}{1.106} & \cellcolor{gray!6}{0.882} & \cellcolor{gray!6}{0.922} & \cellcolor{gray!6}{1.252} & \cellcolor{gray!6}{1.340} & \cellcolor{gray!6}{logh\_scaled} & \cellcolor{gray!6}{intercept}\\
gamma\_3\_1 & -1.767 & -3.194 & -2.986 & -0.826 & -0.382 & cons\_scaled & intercept\\
\cellcolor{gray!6}{gamma\_3\_3} & \cellcolor{gray!6}{1.801} & \cellcolor{gray!6}{0.018} & \cellcolor{gray!6}{0.307} & \cellcolor{gray!6}{2.989} & \cellcolor{gray!6}{3.661} & \cellcolor{gray!6}{cons\_scaled} & \cellcolor{gray!6}{sdmc}\\
\addlinespace
gamma\_3\_4 & -2.055 & -3.684 & -3.390 & -1.050 & -0.514 & cons\_scaled & chl\\
\cellcolor{gray!6}{gamma\_3\_6} & \cellcolor{gray!6}{-1.928} & \cellcolor{gray!6}{-3.624} & \cellcolor{gray!6}{-3.369} & \cellcolor{gray!6}{-0.853} & \cellcolor{gray!6}{-0.260} & \cellcolor{gray!6}{cons\_scaled} & \cellcolor{gray!6}{c\_mass}\\
gamma\_5\_1 & 0.211 & 0.058 & 0.081 & 0.325 & 0.406 & hets\_scaled & intercept\\
\cellcolor{gray!6}{gamma\_7\_1} & \cellcolor{gray!6}{-0.587} & \cellcolor{gray!6}{-1.152} & \cellcolor{gray!6}{-1.048} & \cellcolor{gray!6}{-0.226} & \cellcolor{gray!6}{-0.042} & \cellcolor{gray!6}{rain\_scaled} & \cellcolor{gray!6}{intercept}\\
gamma\_7\_10 & -1.100 & -1.991 & -1.844 & -0.520 & -0.149 & rain\_scaled & log\_sla\\
\addlinespace
\cellcolor{gray!6}{gamma\_7\_2} & \cellcolor{gray!6}{-1.296} & \cellcolor{gray!6}{-2.190} & \cellcolor{gray!6}{-2.055} & \cellcolor{gray!6}{-0.708} & \cellcolor{gray!6}{-0.377} & \cellcolor{gray!6}{rain\_scaled} & \cellcolor{gray!6}{ldmc}\\
gamma\_7\_5 & -1.273 & -1.922 & -1.812 & -0.866 & -0.644 & rain\_scaled & c13\\
\cellcolor{gray!6}{gamma\_8\_11} & \cellcolor{gray!6}{-1.036} & \cellcolor{gray!6}{-1.965} & \cellcolor{gray!6}{-1.796} & \cellcolor{gray!6}{-0.465} & \cellcolor{gray!6}{-0.106} & \cellcolor{gray!6}{cons\_rain} & \cellcolor{gray!6}{log\_lt}\\
gamma\_8\_2 & -1.949 & -3.263 & -3.025 & -1.097 & -0.568 & cons\_rain & ldmc\\
\cellcolor{gray!6}{gamma\_8\_5} & \cellcolor{gray!6}{-1.645} & \cellcolor{gray!6}{-2.643} & \cellcolor{gray!6}{-2.476} & \cellcolor{gray!6}{-0.979} & \cellcolor{gray!6}{-0.615} & \cellcolor{gray!6}{cons\_rain} & \cellcolor{gray!6}{c13}\\
\addlinespace
gamma\_8\_8 & 1.617 & 0.153 & 0.409 & 2.539 & 3.050 & cons\_rain & tlp\\*
\end{longtable}

The columns are as follows:

\begin{itemize}
\tightlist
\item
  \texttt{para}: name of parameters
\item
  \texttt{mean\_}: posterior means
\item
  \texttt{q\_x}: x\% quantile of the posterior distributions. For
  example, \texttt{q2\_5} and \texttt{q5} are 2.5\% and 5\% quantiles,
  respectively.
\item
  \texttt{pred\_name}: individual-level predictors
\item
  \texttt{tarit\_name}: species-level predictors
\end{itemize}

Individual-level predictors are as follows:

\begin{itemize}
\tightlist
\item
  \texttt{int}: intercept
\item
  \texttt{logh\_scaled}: scaled ln height
\item
  \texttt{cons\_scaled}: scaled conspecific seedling densities
\item
  \texttt{hets\_scaled}: scaled heterospecific seedling densities
\item
  \texttt{cona\_scaled\_c}: scaled conspecific tree densities that were
  scaled by \texttt{c=0.27} first
\item
  \texttt{heta\_scaled\_c}: scaled heterospecific tree densities that
  were scaled by \texttt{c=0.27} first
\item
  \texttt{rain\_scaled}: scaled rainfall
\item
  \texttt{cons\_rain}: interaction between \texttt{cons\_scaled} and
  \texttt{rain\_scaled}
\item
  \texttt{cona\_rain}: interaction between \texttt{cona\_scaled} and
  \texttt{rain\_scaled}
\item
  \texttt{hets\_rain}: interaction between \texttt{hets\_scaled} and
  \texttt{rain\_scaled}
\item
  \texttt{heta\_rain}: interaction between \texttt{heta\_scaled} and
  \texttt{rain\_scaled}
\end{itemize}

Species-level predictors are as follows:

\begin{itemize}
\tightlist
\item
  \texttt{ldmc}: leaf dry matter contents
\item
  \texttt{sdmc}: stem dry matter contents
\item
  \texttt{chl}: chlorophyll content
\item
  \texttt{c13}: stable carbon isotope composition,
\item
  \texttt{c\_mass}: carbon concentration
\item
  \texttt{n\_mass}: nitrogen concentration
\item
  \texttt{tlp}: leaf turgor loss point
\item
  \texttt{log\_sla}: natural logarithm of specific leaf area
\item
  \texttt{log\_lt}: natural logarithm of leaf thickness
\end{itemize}

This is the list of \(\gamma_{k,1}\) parameters, which represents
average effects of the each individual-level predictor (e.g.,
\emph{ConS}) across species.

\begin{Shaded}
\begin{Highlighting}[]
\NormalTok{fit9\_gamma }\SpecialCharTok{|\textgreater{}}
  \FunctionTok{filter}\NormalTok{(trait\_name }\SpecialCharTok{==} \StringTok{"intercept"}\NormalTok{) }\SpecialCharTok{|\textgreater{}}
  \FunctionTok{kbl}\NormalTok{(}\AttributeTok{booktabs =} \ConstantTok{TRUE}\NormalTok{, }\AttributeTok{longtable =} \ConstantTok{TRUE}\NormalTok{, }\AttributeTok{format =} \StringTok{"latex"}\NormalTok{) }\SpecialCharTok{|\textgreater{}}
  \FunctionTok{kable\_styling}\NormalTok{(}\AttributeTok{latex\_options =} \FunctionTok{c}\NormalTok{(}\StringTok{"striped"}\NormalTok{, }\StringTok{"repeat\_header"}\NormalTok{))}
\end{Highlighting}
\end{Shaded}

\begin{longtable}[t]{lrrrrrll}
\toprule
para & mean\_ & q2\_5 & q5 & q95 & q97\_5 & pred\_name & trait\_name\\
\midrule
\endfirsthead
\multicolumn{8}{@{}l}{\textit{(continued)}}\\
\toprule
para & mean\_ & q2\_5 & q5 & q95 & q97\_5 & pred\_name & trait\_name\\
\midrule
\endhead

\endfoot
\bottomrule
\endlastfoot
\cellcolor{gray!6}{gamma\_1\_1} & \cellcolor{gray!6}{2.693} & \cellcolor{gray!6}{1.818} & \cellcolor{gray!6}{1.957} & \cellcolor{gray!6}{3.302} & \cellcolor{gray!6}{3.606} & \cellcolor{gray!6}{int} & \cellcolor{gray!6}{intercept}\\
gamma\_2\_1 & 1.106 & 0.882 & 0.922 & 1.252 & 1.340 & logh\_scaled & intercept\\
\cellcolor{gray!6}{gamma\_3\_1} & \cellcolor{gray!6}{-1.767} & \cellcolor{gray!6}{-3.194} & \cellcolor{gray!6}{-2.986} & \cellcolor{gray!6}{-0.826} & \cellcolor{gray!6}{-0.382} & \cellcolor{gray!6}{cons\_scaled} & \cellcolor{gray!6}{intercept}\\
gamma\_4\_1 & -0.083 & -0.627 & -0.535 & 0.271 & 0.455 & cona\_scaled\_c & intercept\\
\cellcolor{gray!6}{gamma\_5\_1} & \cellcolor{gray!6}{0.211} & \cellcolor{gray!6}{0.058} & \cellcolor{gray!6}{0.081} & \cellcolor{gray!6}{0.325} & \cellcolor{gray!6}{0.406} & \cellcolor{gray!6}{hets\_scaled} & \cellcolor{gray!6}{intercept}\\
\addlinespace
gamma\_6\_1 & -0.101 & -0.298 & -0.264 & 0.026 & 0.094 & heta\_scaled\_c & intercept\\
\cellcolor{gray!6}{gamma\_7\_1} & \cellcolor{gray!6}{-0.587} & \cellcolor{gray!6}{-1.152} & \cellcolor{gray!6}{-1.048} & \cellcolor{gray!6}{-0.226} & \cellcolor{gray!6}{-0.042} & \cellcolor{gray!6}{rain\_scaled} & \cellcolor{gray!6}{intercept}\\
gamma\_8\_1 & -0.624 & -1.364 & -1.230 & -0.165 & 0.106 & cons\_rain & intercept\\
\cellcolor{gray!6}{gamma\_9\_1} & \cellcolor{gray!6}{-0.238} & \cellcolor{gray!6}{-0.512} & \cellcolor{gray!6}{-0.462} & \cellcolor{gray!6}{-0.068} & \cellcolor{gray!6}{0.019} & \cellcolor{gray!6}{cona\_rain} & \cellcolor{gray!6}{intercept}\\
gamma\_10\_1 & -0.016 & -0.151 & -0.127 & 0.066 & 0.109 & hets\_rain & intercept\\
\addlinespace
\cellcolor{gray!6}{gamma\_11\_1} & \cellcolor{gray!6}{-0.014} & \cellcolor{gray!6}{-0.145} & \cellcolor{gray!6}{-0.126} & \cellcolor{gray!6}{0.073} & \cellcolor{gray!6}{0.121} & \cellcolor{gray!6}{heta\_rain} & \cellcolor{gray!6}{intercept}\\*
\end{longtable}

\hypertarget{gamma-coefficients-for-rainy-seasons}{%
\subsection{\texorpdfstring{\(\gamma\) coefficients for rainy
seasons}{\textbackslash gamma coefficients for rainy seasons}}\label{gamma-coefficients-for-rainy-seasons}}

\begin{itemize}
\tightlist
\item
  \texttt{fit10\_gamma}: \(\gamma\) values of all the parameters for
  rainy seasons
\end{itemize}

\begin{Shaded}
\begin{Highlighting}[]
\NormalTok{targets}\SpecialCharTok{::}\FunctionTok{tar\_load}\NormalTok{(fit10\_gamma)}
\NormalTok{fit10\_gamma[}\DecValTok{2}\SpecialCharTok{:}\DecValTok{6}\NormalTok{] }\OtherTok{\textless{}{-}} \FunctionTok{round}\NormalTok{(fit10\_gamma[}\DecValTok{2}\SpecialCharTok{:}\DecValTok{6}\NormalTok{], }\DecValTok{3}\NormalTok{)}
\end{Highlighting}
\end{Shaded}

\begin{itemize}
\tightlist
\item
  \texttt{n\_mass}: nitrogen concentration
\end{itemize}

Other details are same as in the tables above except for
\texttt{c=0.24}. The full list is available as Table S2 in a separate
file.

\begin{Shaded}
\begin{Highlighting}[]
\NormalTok{fit10\_gamma }\SpecialCharTok{|\textgreater{}}
  \FunctionTok{filter}\NormalTok{(q2\_5 }\SpecialCharTok{*}\NormalTok{ q97\_5 }\SpecialCharTok{\textgreater{}} \DecValTok{0}\NormalTok{) }\SpecialCharTok{|\textgreater{}}
  \FunctionTok{arrange}\NormalTok{(para) }\SpecialCharTok{|\textgreater{}}
  \FunctionTok{kbl}\NormalTok{(}\AttributeTok{booktabs =} \ConstantTok{TRUE}\NormalTok{, }\AttributeTok{longtable =} \ConstantTok{TRUE}\NormalTok{, }\AttributeTok{format =} \StringTok{"latex"}\NormalTok{) }\SpecialCharTok{|\textgreater{}}
  \FunctionTok{kable\_styling}\NormalTok{(}\AttributeTok{latex\_options =} \FunctionTok{c}\NormalTok{(}\StringTok{"striped"}\NormalTok{, }\StringTok{"repeat\_header"}\NormalTok{))}
\end{Highlighting}
\end{Shaded}

\begin{longtable}[t]{lrrrrrll}
\toprule
para & mean\_ & q2\_5 & q5 & q95 & q97\_5 & pred\_name & trait\_name\\
\midrule
\endfirsthead
\multicolumn{8}{@{}l}{\textit{(continued)}}\\
\toprule
para & mean\_ & q2\_5 & q5 & q95 & q97\_5 & pred\_name & trait\_name\\
\midrule
\endhead

\endfoot
\bottomrule
\endlastfoot
\cellcolor{gray!6}{gamma\_1\_1} & \cellcolor{gray!6}{2.811} & \cellcolor{gray!6}{1.897} & \cellcolor{gray!6}{2.038} & \cellcolor{gray!6}{3.414} & \cellcolor{gray!6}{3.781} & \cellcolor{gray!6}{int} & \cellcolor{gray!6}{intercept}\\
gamma\_2\_1 & 1.232 & 1.001 & 1.040 & 1.384 & 1.475 & logh\_scaled & intercept\\
\cellcolor{gray!6}{gamma\_4\_1} & \cellcolor{gray!6}{-0.687} & \cellcolor{gray!6}{-1.231} & \cellcolor{gray!6}{-1.124} & \cellcolor{gray!6}{-0.349} & \cellcolor{gray!6}{-0.170} & \cellcolor{gray!6}{cona\_scaled\_c} & \cellcolor{gray!6}{intercept}\\
gamma\_5\_1 & 0.176 & 0.035 & 0.057 & 0.272 & 0.328 & hets\_scaled & intercept\\
\cellcolor{gray!6}{gamma\_7\_7} & \cellcolor{gray!6}{1.489} & \cellcolor{gray!6}{0.219} & \cellcolor{gray!6}{0.428} & \cellcolor{gray!6}{2.303} & \cellcolor{gray!6}{2.727} & \cellcolor{gray!6}{rain\_scaled} & \cellcolor{gray!6}{n\_mass}\\
\addlinespace
gamma\_7\_8 & 2.290 & 1.102 & 1.257 & 3.099 & 3.536 & rain\_scaled & tlp\\
\cellcolor{gray!6}{gamma\_8\_7} & \cellcolor{gray!6}{3.245} & \cellcolor{gray!6}{0.642} & \cellcolor{gray!6}{1.070} & \cellcolor{gray!6}{4.941} & \cellcolor{gray!6}{5.819} & \cellcolor{gray!6}{cons\_rain} & \cellcolor{gray!6}{n\_mass}\\
gamma\_8\_8 & 4.552 & 2.000 & 2.412 & 6.249 & 7.173 & cons\_rain & tlp\\*
\end{longtable}

This is the list of \(\gamma_{k,1}\) parameters.

\begin{Shaded}
\begin{Highlighting}[]
\NormalTok{fit10\_gamma }\SpecialCharTok{|\textgreater{}}
  \FunctionTok{filter}\NormalTok{(trait\_name }\SpecialCharTok{==} \StringTok{"intercept"}\NormalTok{) }\SpecialCharTok{|\textgreater{}}
  \FunctionTok{kbl}\NormalTok{(}\AttributeTok{booktabs =} \ConstantTok{TRUE}\NormalTok{, }\AttributeTok{longtable =} \ConstantTok{TRUE}\NormalTok{, }\AttributeTok{format =} \StringTok{"latex"}\NormalTok{) }\SpecialCharTok{|\textgreater{}}
  \FunctionTok{kable\_styling}\NormalTok{(}\AttributeTok{latex\_options =} \FunctionTok{c}\NormalTok{(}\StringTok{"striped"}\NormalTok{, }\StringTok{"repeat\_header"}\NormalTok{))}
\end{Highlighting}
\end{Shaded}

\begin{longtable}[t]{lrrrrrll}
\toprule
para & mean\_ & q2\_5 & q5 & q95 & q97\_5 & pred\_name & trait\_name\\
\midrule
\endfirsthead
\multicolumn{8}{@{}l}{\textit{(continued)}}\\
\toprule
para & mean\_ & q2\_5 & q5 & q95 & q97\_5 & pred\_name & trait\_name\\
\midrule
\endhead

\endfoot
\bottomrule
\endlastfoot
\cellcolor{gray!6}{gamma\_1\_1} & \cellcolor{gray!6}{2.811} & \cellcolor{gray!6}{1.897} & \cellcolor{gray!6}{2.038} & \cellcolor{gray!6}{3.414} & \cellcolor{gray!6}{3.781} & \cellcolor{gray!6}{int} & \cellcolor{gray!6}{intercept}\\
gamma\_2\_1 & 1.232 & 1.001 & 1.040 & 1.384 & 1.475 & logh\_scaled & intercept\\
\cellcolor{gray!6}{gamma\_3\_1} & \cellcolor{gray!6}{-0.802} & \cellcolor{gray!6}{-2.499} & \cellcolor{gray!6}{-2.156} & \cellcolor{gray!6}{0.171} & \cellcolor{gray!6}{0.679} & \cellcolor{gray!6}{cons\_scaled} & \cellcolor{gray!6}{intercept}\\
gamma\_4\_1 & -0.687 & -1.231 & -1.124 & -0.349 & -0.170 & cona\_scaled\_c & intercept\\
\cellcolor{gray!6}{gamma\_5\_1} & \cellcolor{gray!6}{0.176} & \cellcolor{gray!6}{0.035} & \cellcolor{gray!6}{0.057} & \cellcolor{gray!6}{0.272} & \cellcolor{gray!6}{0.328} & \cellcolor{gray!6}{hets\_scaled} & \cellcolor{gray!6}{intercept}\\
\addlinespace
gamma\_6\_1 & 0.014 & -0.252 & -0.201 & 0.173 & 0.260 & heta\_scaled\_c & intercept\\
\cellcolor{gray!6}{gamma\_7\_1} & \cellcolor{gray!6}{0.317} & \cellcolor{gray!6}{-0.339} & \cellcolor{gray!6}{-0.233} & \cellcolor{gray!6}{0.743} & \cellcolor{gray!6}{0.979} & \cellcolor{gray!6}{rain\_scaled} & \cellcolor{gray!6}{intercept}\\
gamma\_8\_1 & 0.236 & -0.698 & -0.566 & 0.877 & 1.245 & cons\_rain & intercept\\
\cellcolor{gray!6}{gamma\_9\_1} & \cellcolor{gray!6}{0.060} & \cellcolor{gray!6}{-0.249} & \cellcolor{gray!6}{-0.195} & \cellcolor{gray!6}{0.262} & \cellcolor{gray!6}{0.362} & \cellcolor{gray!6}{cona\_rain} & \cellcolor{gray!6}{intercept}\\
gamma\_10\_1 & 0.003 & -0.113 & -0.094 & 0.078 & 0.120 & hets\_rain & intercept\\
\addlinespace
\cellcolor{gray!6}{gamma\_11\_1} & \cellcolor{gray!6}{0.002} & \cellcolor{gray!6}{-0.167} & \cellcolor{gray!6}{-0.137} & \cellcolor{gray!6}{0.109} & \cellcolor{gray!6}{0.177} & \cellcolor{gray!6}{heta\_rain} & \cellcolor{gray!6}{intercept}\\*
\end{longtable}

\hypertarget{r-session-information}{%
\section{R session information}\label{r-session-information}}

\begin{Shaded}
\begin{Highlighting}[]
\NormalTok{devtools}\SpecialCharTok{::}\FunctionTok{session\_info}\NormalTok{()}
\CommentTok{\#\textgreater{} {-} Session info {-}{-}{-}{-}{-}{-}{-}{-}{-}{-}{-}{-}{-}{-}{-}{-}{-}{-}{-}{-}{-}{-}{-}{-}{-}{-}{-}{-}{-}{-}{-}{-}{-}{-}{-}{-}{-}{-}{-}{-}{-}{-}{-}{-}{-}{-}{-}{-}{-}{-}{-}{-}{-}{-}{-}{-}{-}{-}{-}{-}{-}{-}{-}}
\CommentTok{\#\textgreater{}  setting  value}
\CommentTok{\#\textgreater{}  version  R version 4.2.1 (2022{-}06{-}23)}
\CommentTok{\#\textgreater{}  os       Ubuntu 20.04.4 LTS}
\CommentTok{\#\textgreater{}  system   x86\_64, linux{-}gnu}
\CommentTok{\#\textgreater{}  ui       X11}
\CommentTok{\#\textgreater{}  language (EN)}
\CommentTok{\#\textgreater{}  collate  en\_US.UTF{-}8}
\CommentTok{\#\textgreater{}  ctype    en\_US.UTF{-}8}
\CommentTok{\#\textgreater{}  tz       Etc/UTC}
\CommentTok{\#\textgreater{}  date     2022{-}08{-}11}
\CommentTok{\#\textgreater{}  pandoc   2.18 @ /usr/local/bin/ (via rmarkdown)}
\CommentTok{\#\textgreater{} }
\CommentTok{\#\textgreater{} {-} Packages {-}{-}{-}{-}{-}{-}{-}{-}{-}{-}{-}{-}{-}{-}{-}{-}{-}{-}{-}{-}{-}{-}{-}{-}{-}{-}{-}{-}{-}{-}{-}{-}{-}{-}{-}{-}{-}{-}{-}{-}{-}{-}{-}{-}{-}{-}{-}{-}{-}{-}{-}{-}{-}{-}{-}{-}{-}{-}{-}{-}{-}{-}{-}{-}{-}{-}{-}}
\CommentTok{\#\textgreater{}  ! package     * version date (UTC) lib source}
\CommentTok{\#\textgreater{}  P assertthat    0.2.1   2019{-}03{-}21 [?] RSPM (R 4.2.0)}
\CommentTok{\#\textgreater{}  P backports     1.4.1   2021{-}12{-}13 [?] RSPM (R 4.2.0)}
\CommentTok{\#\textgreater{}  P base64url     1.4     2018{-}05{-}14 [?] RSPM (R 4.2.0)}
\CommentTok{\#\textgreater{}  P broom         1.0.0   2022{-}07{-}01 [?] RSPM (R 4.2.0)}
\CommentTok{\#\textgreater{}  P cachem        1.0.6   2021{-}08{-}19 [?] RSPM (R 4.2.0)}
\CommentTok{\#\textgreater{}  P callr         3.7.0   2021{-}04{-}20 [?] RSPM (R 4.2.0)}
\CommentTok{\#\textgreater{}  P cellranger    1.1.0   2016{-}07{-}27 [?] RSPM (R 4.2.0)}
\CommentTok{\#\textgreater{}  P cli           3.3.0   2022{-}04{-}25 [?] RSPM (R 4.2.0)}
\CommentTok{\#\textgreater{}    codetools     0.2{-}18  2020{-}11{-}04 [2] CRAN (R 4.2.1)}
\CommentTok{\#\textgreater{}  P colorspace    2.0{-}3   2022{-}02{-}21 [?] RSPM (R 4.2.0)}
\CommentTok{\#\textgreater{}  P crayon        1.5.1   2022{-}03{-}26 [?] RSPM (R 4.2.0)}
\CommentTok{\#\textgreater{}  P data.table    1.14.2  2021{-}09{-}27 [?] RSPM (R 4.2.0)}
\CommentTok{\#\textgreater{}  P DBI           1.1.3   2022{-}06{-}18 [?] RSPM (R 4.2.0)}
\CommentTok{\#\textgreater{}  P dbplyr        2.2.1   2022{-}06{-}27 [?] RSPM (R 4.2.0)}
\CommentTok{\#\textgreater{}  P devtools      2.4.3   2021{-}11{-}30 [?] RSPM (R 4.2.0)}
\CommentTok{\#\textgreater{}  P digest        0.6.29  2021{-}12{-}01 [?] RSPM (R 4.2.0)}
\CommentTok{\#\textgreater{}  P dplyr       * 1.0.9   2022{-}04{-}28 [?] RSPM (R 4.2.0)}
\CommentTok{\#\textgreater{}  P ellipsis      0.3.2   2021{-}04{-}29 [?] RSPM (R 4.2.0)}
\CommentTok{\#\textgreater{}  P evaluate      0.15    2022{-}02{-}18 [?] RSPM (R 4.2.0)}
\CommentTok{\#\textgreater{}  P fansi         1.0.3   2022{-}03{-}24 [?] RSPM (R 4.2.0)}
\CommentTok{\#\textgreater{}  P fastmap       1.1.0   2021{-}01{-}25 [?] RSPM (R 4.2.0)}
\CommentTok{\#\textgreater{}  P forcats     * 0.5.1   2021{-}01{-}27 [?] RSPM (R 4.2.0)}
\CommentTok{\#\textgreater{}  P fs            1.5.2   2021{-}12{-}08 [?] RSPM (R 4.2.0)}
\CommentTok{\#\textgreater{}    fst           0.9.8   2022{-}02{-}08 [1] RSPM (R 4.2.1)}
\CommentTok{\#\textgreater{}    fstcore     * 0.9.12  2022{-}03{-}23 [1] RSPM (R 4.2.1)}
\CommentTok{\#\textgreater{}  P generics      0.1.2   2022{-}01{-}31 [?] RSPM (R 4.2.0)}
\CommentTok{\#\textgreater{}  P ggplot2     * 3.3.6   2022{-}05{-}03 [?] RSPM (R 4.2.0)}
\CommentTok{\#\textgreater{}  P glue          1.6.2   2022{-}02{-}24 [?] RSPM (R 4.2.0)}
\CommentTok{\#\textgreater{}  P gtable        0.3.0   2019{-}03{-}25 [?] RSPM (R 4.2.0)}
\CommentTok{\#\textgreater{}  P haven         2.5.0   2022{-}04{-}15 [?] RSPM (R 4.2.0)}
\CommentTok{\#\textgreater{}  P here        * 1.0.1   2020{-}12{-}13 [?] RSPM (R 4.2.0)}
\CommentTok{\#\textgreater{}  P hms           1.1.1   2021{-}09{-}26 [?] RSPM (R 4.2.0)}
\CommentTok{\#\textgreater{}  P htmltools     0.5.2   2021{-}08{-}25 [?] RSPM (R 4.2.0)}
\CommentTok{\#\textgreater{}  P httr          1.4.3   2022{-}05{-}04 [?] RSPM (R 4.2.0)}
\CommentTok{\#\textgreater{}  P igraph        1.3.2   2022{-}06{-}13 [?] RSPM (R 4.2.0)}
\CommentTok{\#\textgreater{}  P jsonlite      1.8.0   2022{-}02{-}22 [?] RSPM (R 4.2.0)}
\CommentTok{\#\textgreater{}  P kableExtra  * 1.3.4   2021{-}02{-}20 [?] RSPM (R 4.2.0)}
\CommentTok{\#\textgreater{}  P knitr       * 1.39    2022{-}04{-}26 [?] RSPM (R 4.2.0)}
\CommentTok{\#\textgreater{}  P lifecycle     1.0.1   2021{-}09{-}24 [?] RSPM (R 4.2.0)}
\CommentTok{\#\textgreater{}  P lubridate     1.8.0   2021{-}10{-}07 [?] RSPM (R 4.2.0)}
\CommentTok{\#\textgreater{}  P magrittr      2.0.3   2022{-}03{-}30 [?] RSPM (R 4.2.0)}
\CommentTok{\#\textgreater{}  P memoise       2.0.1   2021{-}11{-}26 [?] RSPM (R 4.2.0)}
\CommentTok{\#\textgreater{}  P modelr        0.1.8   2020{-}05{-}19 [?] RSPM (R 4.2.0)}
\CommentTok{\#\textgreater{}  P munsell       0.5.0   2018{-}06{-}12 [?] RSPM (R 4.2.0)}
\CommentTok{\#\textgreater{}  P pillar        1.7.0   2022{-}02{-}01 [?] RSPM (R 4.2.0)}
\CommentTok{\#\textgreater{}  P pkgbuild      1.3.1   2021{-}12{-}20 [?] RSPM (R 4.2.0)}
\CommentTok{\#\textgreater{}  P pkgconfig     2.0.3   2019{-}09{-}22 [?] RSPM (R 4.2.0)}
\CommentTok{\#\textgreater{}  P pkgload       1.3.0   2022{-}06{-}27 [?] RSPM (R 4.2.0)}
\CommentTok{\#\textgreater{}  P prettyunits   1.1.1   2020{-}01{-}24 [?] RSPM (R 4.2.0)}
\CommentTok{\#\textgreater{}  P processx      3.6.1   2022{-}06{-}17 [?] RSPM (R 4.2.0)}
\CommentTok{\#\textgreater{}  P ps            1.7.1   2022{-}06{-}18 [?] RSPM (R 4.2.0)}
\CommentTok{\#\textgreater{}  P purrr       * 0.3.4   2020{-}04{-}17 [?] RSPM (R 4.2.0)}
\CommentTok{\#\textgreater{}  P R6            2.5.1   2021{-}08{-}19 [?] RSPM (R 4.2.0)}
\CommentTok{\#\textgreater{}  P Rcpp          1.0.8.3 2022{-}03{-}17 [?] RSPM (R 4.2.0)}
\CommentTok{\#\textgreater{}  P readr       * 2.1.2   2022{-}01{-}30 [?] RSPM (R 4.2.0)}
\CommentTok{\#\textgreater{}  P readxl        1.4.0   2022{-}03{-}28 [?] RSPM (R 4.2.0)}
\CommentTok{\#\textgreater{}  P remotes       2.4.2   2021{-}11{-}30 [?] RSPM (R 4.2.0)}
\CommentTok{\#\textgreater{}    renv          0.15.5  2022{-}07{-}05 [1] Github (rstudio/renv@085673d)}
\CommentTok{\#\textgreater{}  P reprex        2.0.1   2021{-}08{-}05 [?] RSPM (R 4.2.0)}
\CommentTok{\#\textgreater{}  P rlang         1.0.3   2022{-}06{-}27 [?] RSPM (R 4.2.0)}
\CommentTok{\#\textgreater{}  P rmarkdown     2.14    2022{-}04{-}25 [?] RSPM (R 4.2.0)}
\CommentTok{\#\textgreater{}  P rprojroot     2.0.3   2022{-}04{-}02 [?] RSPM (R 4.2.0)}
\CommentTok{\#\textgreater{}  P rstudioapi    0.13    2020{-}11{-}12 [?] RSPM (R 4.2.0)}
\CommentTok{\#\textgreater{}  P rvest         1.0.2   2021{-}10{-}16 [?] RSPM (R 4.2.0)}
\CommentTok{\#\textgreater{}  P scales        1.2.0   2022{-}04{-}13 [?] RSPM (R 4.2.0)}
\CommentTok{\#\textgreater{}  P sessioninfo   1.2.2   2021{-}12{-}06 [?] RSPM (R 4.2.0)}
\CommentTok{\#\textgreater{}  P stringi       1.7.6   2021{-}11{-}29 [?] RSPM (R 4.2.0)}
\CommentTok{\#\textgreater{}  P stringr     * 1.4.0   2019{-}02{-}10 [?] RSPM (R 4.2.0)}
\CommentTok{\#\textgreater{}  P svglite       2.1.0   2022{-}02{-}03 [?] RSPM (R 4.2.0)}
\CommentTok{\#\textgreater{}  P systemfonts   1.0.4   2022{-}02{-}11 [?] RSPM (R 4.2.0)}
\CommentTok{\#\textgreater{}  P targets       0.12.1  2022{-}06{-}03 [?] RSPM (R 4.2.0)}
\CommentTok{\#\textgreater{}  P tibble      * 3.1.7   2022{-}05{-}03 [?] RSPM (R 4.2.0)}
\CommentTok{\#\textgreater{}  P tidyr       * 1.2.0   2022{-}02{-}01 [?] RSPM (R 4.2.0)}
\CommentTok{\#\textgreater{}  P tidyselect    1.1.2   2022{-}02{-}21 [?] RSPM (R 4.2.0)}
\CommentTok{\#\textgreater{}  P tidyverse   * 1.3.1   2021{-}04{-}15 [?] RSPM (R 4.2.0)}
\CommentTok{\#\textgreater{}  P tzdb          0.3.0   2022{-}03{-}28 [?] RSPM (R 4.2.0)}
\CommentTok{\#\textgreater{}  P usethis       2.1.6   2022{-}05{-}25 [?] RSPM (R 4.2.0)}
\CommentTok{\#\textgreater{}  P utf8          1.2.2   2021{-}07{-}24 [?] RSPM (R 4.2.0)}
\CommentTok{\#\textgreater{}  P vctrs         0.4.1   2022{-}04{-}13 [?] RSPM (R 4.2.0)}
\CommentTok{\#\textgreater{}  P viridisLite   0.4.0   2021{-}04{-}13 [?] RSPM (R 4.2.0)}
\CommentTok{\#\textgreater{}  P webshot       0.5.3   2022{-}04{-}14 [?] RSPM (R 4.2.0)}
\CommentTok{\#\textgreater{}  P withr         2.5.0   2022{-}03{-}03 [?] RSPM (R 4.2.0)}
\CommentTok{\#\textgreater{}  P xfun          0.31    2022{-}05{-}10 [?] RSPM (R 4.2.0)}
\CommentTok{\#\textgreater{}  P xml2          1.3.3   2021{-}11{-}30 [?] RSPM (R 4.2.0)}
\CommentTok{\#\textgreater{}  P yaml          2.3.5   2022{-}02{-}21 [?] RSPM (R 4.2.0)}
\CommentTok{\#\textgreater{} }
\CommentTok{\#\textgreater{}  [1] /home/mattocci/seedling{-}stan/renv/library/linux{-}ubuntu{-}focal/R{-}4.2/x86\_64{-}pc{-}linux{-}gnu}
\CommentTok{\#\textgreater{}  [2] /usr/local/lib/R/library}
\CommentTok{\#\textgreater{} }
\CommentTok{\#\textgreater{}  P {-}{-} Loaded and on{-}disk path mismatch.}
\CommentTok{\#\textgreater{} }
\CommentTok{\#\textgreater{} {-}{-}{-}{-}{-}{-}{-}{-}{-}{-}{-}{-}{-}{-}{-}{-}{-}{-}{-}{-}{-}{-}{-}{-}{-}{-}{-}{-}{-}{-}{-}{-}{-}{-}{-}{-}{-}{-}{-}{-}{-}{-}{-}{-}{-}{-}{-}{-}{-}{-}{-}{-}{-}{-}{-}{-}{-}{-}{-}{-}{-}{-}{-}{-}{-}{-}{-}{-}{-}{-}{-}{-}{-}{-}{-}{-}{-}{-}}
\end{Highlighting}
\end{Shaded}

\hypertarget{references}{%
\section*{References}\label{references}}
\addcontentsline{toc}{section}{References}

\hypertarget{refs}{}
\begin{CSLReferences}{1}{0}
\leavevmode\vadjust pre{\hypertarget{ref-Betancourt2016}{}}%
Betancourt, M. (2016).
\href{https://doi.org/10.48550/arXiv.1604.00695}{Diagnosing {Suboptimal
Cotangent Disintegrations} in {Hamiltonian Monte Carlo}}. \emph{arXiv}.

\leavevmode\vadjust pre{\hypertarget{ref-Carpenter2017}{}}%
Carpenter, B., Gelman, A., Hoffman, M.D., Lee, D., Goodrich, B.,
Betancourt, M., \emph{et al.} (2017).
\href{https://doi.org/10.18637/jss.v076.i01}{Stan : {A Probabilistic
Programming Language}}. \emph{Journal of Statistical Software}, 76,
1--32.

\leavevmode\vadjust pre{\hypertarget{ref-Gelman2013}{}}%
Gelman, A., Carlin, J.B., Stern, H.S., Dunson, D.B., Vehtari, A. \&
Rubin, D.B. (2013). \emph{Bayesian {Data Analysis}, {Third Edition}}.
{Chapman \& Hall/CRC}, {Boca Raton, FL, USA.}

\leavevmode\vadjust pre{\hypertarget{ref-StanDevelopmentTeam2022}{}}%
Stan Development Team. (2022). Stan {Modeling Language Users Guide} and
{Reference Manual}, v2.29.2.

\end{CSLReferences}



\end{document}
