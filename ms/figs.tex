% Options for packages loaded elsewhere
\PassOptionsToPackage{unicode}{hyperref}
\PassOptionsToPackage{hyphens}{url}
\PassOptionsToPackage{dvipsnames,svgnames,x11names}{xcolor}
%
\documentclass[
  12pt,
  letterpaper,
  DIV=11,
  numbers=noendperiod]{scrartcl}

\usepackage{amsmath,amssymb}
\usepackage{lmodern}
\usepackage{iftex}
\ifPDFTeX
  \usepackage[T1]{fontenc}
  \usepackage[utf8]{inputenc}
  \usepackage{textcomp} % provide euro and other symbols
\else % if luatex or xetex
  \usepackage{unicode-math}
  \defaultfontfeatures{Scale=MatchLowercase}
  \defaultfontfeatures[\rmfamily]{Ligatures=TeX,Scale=1}
\fi
% Use upquote if available, for straight quotes in verbatim environments
\IfFileExists{upquote.sty}{\usepackage{upquote}}{}
\IfFileExists{microtype.sty}{% use microtype if available
  \usepackage[]{microtype}
  \UseMicrotypeSet[protrusion]{basicmath} % disable protrusion for tt fonts
}{}
\makeatletter
\@ifundefined{KOMAClassName}{% if non-KOMA class
  \IfFileExists{parskip.sty}{%
    \usepackage{parskip}
  }{% else
    \setlength{\parindent}{0pt}
    \setlength{\parskip}{6pt plus 2pt minus 1pt}}
}{% if KOMA class
  \KOMAoptions{parskip=half}}
\makeatother
\usepackage{xcolor}
\usepackage[margin=1in]{geometry}
\setlength{\emergencystretch}{3em} % prevent overfull lines
\setcounter{secnumdepth}{-\maxdimen} % remove section numbering
% Make \paragraph and \subparagraph free-standing
\ifx\paragraph\undefined\else
  \let\oldparagraph\paragraph
  \renewcommand{\paragraph}[1]{\oldparagraph{#1}\mbox{}}
\fi
\ifx\subparagraph\undefined\else
  \let\oldsubparagraph\subparagraph
  \renewcommand{\subparagraph}[1]{\oldsubparagraph{#1}\mbox{}}
\fi


\providecommand{\tightlist}{%
  \setlength{\itemsep}{0pt}\setlength{\parskip}{0pt}}\usepackage{longtable,booktabs,array}
\usepackage{calc} % for calculating minipage widths
% Correct order of tables after \paragraph or \subparagraph
\usepackage{etoolbox}
\makeatletter
\patchcmd\longtable{\par}{\if@noskipsec\mbox{}\fi\par}{}{}
\makeatother
% Allow footnotes in longtable head/foot
\IfFileExists{footnotehyper.sty}{\usepackage{footnotehyper}}{\usepackage{footnote}}
\makesavenoteenv{longtable}
\usepackage{graphicx}
\makeatletter
\def\maxwidth{\ifdim\Gin@nat@width>\linewidth\linewidth\else\Gin@nat@width\fi}
\def\maxheight{\ifdim\Gin@nat@height>\textheight\textheight\else\Gin@nat@height\fi}
\makeatother
% Scale images if necessary, so that they will not overflow the page
% margins by default, and it is still possible to overwrite the defaults
% using explicit options in \includegraphics[width, height, ...]{}
\setkeys{Gin}{width=\maxwidth,height=\maxheight,keepaspectratio}
% Set default figure placement to htbp
\makeatletter
\def\fps@figure{htbp}
\makeatother

\usepackage{xr}
\usepackage[default]{sourcesanspro}
\usepackage{sourcecodepro}
\usepackage{lineno}
\linenumbers
\KOMAoption{captions}{tableheading}
\makeatletter
\makeatother
\makeatletter
\makeatother
\makeatletter
\@ifpackageloaded{caption}{}{\usepackage{caption}}
\AtBeginDocument{%
\ifdefined\contentsname
  \renewcommand*\contentsname{Table of contents}
\else
  \newcommand\contentsname{Table of contents}
\fi
\ifdefined\listfigurename
  \renewcommand*\listfigurename{List of Figures}
\else
  \newcommand\listfigurename{List of Figures}
\fi
\ifdefined\listtablename
  \renewcommand*\listtablename{List of Tables}
\else
  \newcommand\listtablename{List of Tables}
\fi
\ifdefined\figurename
  \renewcommand*\figurename{Fig.}
\else
  \newcommand\figurename{Fig.}
\fi
\ifdefined\tablename
  \renewcommand*\tablename{Table}
\else
  \newcommand\tablename{Table}
\fi
}
\@ifpackageloaded{float}{}{\usepackage{float}}
\floatstyle{ruled}
\@ifundefined{c@chapter}{\newfloat{codelisting}{h}{lop}}{\newfloat{codelisting}{h}{lop}[chapter]}
\floatname{codelisting}{Listing}
\newcommand*\listoflistings{\listof{codelisting}{List of Listings}}
\makeatother
\makeatletter
\@ifpackageloaded{caption}{}{\usepackage{caption}}
\@ifpackageloaded{subcaption}{}{\usepackage{subcaption}}
\makeatother
\makeatletter
\@ifpackageloaded{tcolorbox}{}{\usepackage[many]{tcolorbox}}
\makeatother
\makeatletter
\@ifundefined{shadecolor}{\definecolor{shadecolor}{rgb}{.97, .97, .97}}
\makeatother
\makeatletter
\makeatother
\ifLuaTeX
  \usepackage{selnolig}  % disable illegal ligatures
\fi
\IfFileExists{bookmark.sty}{\usepackage{bookmark}}{\usepackage{hyperref}}
\IfFileExists{xurl.sty}{\usepackage{xurl}}{} % add URL line breaks if available
\urlstyle{same} % disable monospaced font for URLs
\hypersetup{
  colorlinks=true,
  linkcolor={blue},
  filecolor={Maroon},
  citecolor={Blue},
  urlcolor={Blue},
  pdfcreator={LaTeX via pandoc}}

\author{}
\date{}

\begin{document}
\ifdefined\Shaded\renewenvironment{Shaded}{\begin{tcolorbox}[boxrule=0pt, sharp corners, interior hidden, borderline west={3pt}{0pt}{shadecolor}, enhanced, breakable, frame hidden]}{\end{tcolorbox}}\fi

\textbf{Figure 1} Standardized regression coefficients common to all
species (\(\gamma_{k,1}\)) modeling the effects of log of individual
height (ln Height), conspecific seedling density (ConS), conspecific
tree density (ConT), heterospecific seedling density (HetS),
heterospecific tree density (HetS), rainfall, and interactions between
densities and rainfall for dry and rainy seasons. Thick and thin lines
indicate 90\% and 95\% credible intervals, respectively. Circles show
posterior means of coefficients. Filled circles indicate significant
effects and open circles indicate non-significance effects. Positive
\(\gamma_{k,1}\) values indicate higher survival rates with increasing
values of the predictors, while negative \(\gamma_{k,1}\) values
indicate lower survival rates with increasing values of the predictors.
Note that predictors are scaled to a mean of 0 andstandard deviation of
1 within each season.

\includegraphics{../figs/coef_trait_int_s.pdf}

\newpage

\textbf{Figure 2} Relatonships between species traits and
individual-level predictors on survival rates in dry seasons. Points and
bars indicate posteriors means and 95\% credible intervals (CIs) of
coefficients for each species (\(\beta_{k,j}\)), respectively. Slopes
indicate the relative influence of functional traits on the response of
species survival to each individual-level predictors (\(\gamma_{k,l}\)).
The 95\% CIs are presented as the shaded area. Note that the effecs of
traits of not interest are controled for the slopes (\(\gamma_{k,l}\))
but not controlled for the points (\(\beta_{k,j}\)). A-C: the
species-specific response to conspecific seedling density (ConS;
\(\beta_{3,j}\)) and its relationship with functional (SDMC: stem dry
matter content; C: carbon concertration; Chlorophyll: leaf chlorophyll
content); D-F: the species-specific response to rainfall
(\(\beta_{7,j}\)) and its relationship with functional traits (LDMC:
leaf dry matter content, \(\delta\)C\textsubscript{13}: stable carbon
isotope composition, \(\pi\)\textsubscript{tlp}: leaf turgor loss
point); G-J: the species-specific response to the interaction between
ConS and rainfall (\(\beta_{8,j}\)) and its relationship with functional
traits (LDMC, \(\pi\)\textsubscript{tlp}, \(\delta\)C\textsubscript{13},
ln LT: log of leaf thickness).

\includegraphics{../figs/beta_dry.pdf}

\newpage

\textbf{Figure 3} Relatonships between species traits and
individual-level predictors on survival rates in rainy seasons. Details
as for Figure 2. A-B: the species-specific response to rainfall
(\(\beta_{7,j}\)) and its relationship with functional traits (N:
nitrogen concertraiont, \(\pi\)\textsubscript{tlp}: leaf turgor loss
point); C-D: the species-specific response to the interaction between
conspecific seedling density (ConS) and rainfall (\(\beta_{8,j}\)) and
its relationship with functional traits (N, \(\pi\)\textsubscript{tlp}).

\includegraphics{../figs/beta_wet.pdf}



\end{document}
